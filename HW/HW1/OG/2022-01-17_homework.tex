\documentclass[12pt]{article}
\usepackage{amscd,amssymb,amsthm,amsxtra,exscale,latexsym,verbatim,paralist}
\usepackage{mathrsfs}
\usepackage[T1]{fontenc}
\usepackage{newtxmath,newtxtext}
\usepackage[margin=1in]{geometry}

%\usepackage{mathtools}
%\usepackage{multicol}
\usepackage{tikz}

\pagestyle{empty} 
\setlength{\parindent}{0pt} 
\setlength{\parskip}{\baselineskip}

\theoremstyle{plain}
\newtheorem{ex}{Exercise}

\renewcommand{\proofname}{Solution}

%\makeatletter
%\renewcommand*\env@matrix[1][*\c@MaxMatrixCols c]{%
%  \hskip -\arraycolsep
%  \let\@ifnextchar\new@ifnextchar
%  \array{#1}}
%\makeatother

\begin{document}

MTH 385 \qquad Homework due 2022-01-17

\begin{ex} [1.2.3]
  Show that any integer square leaves remainder $0$ or $1$ on division by $4$.
\end{ex}

\begin{proof}
 
\end{proof}

\begin{ex} [1.2.4]
  Deduce from Exercise~1.2.3 that if $(a,b,c)$ is a Pythagorean triple then $a$ and $b$ cannot both be odd.
\end{ex}

\begin{proof}
 
\end{proof}

The parameter $t$ in the pair $\left(\frac{1-t^2}{1+t^2},\frac{2t}{1+t^2}\right)$ runs through all rational numbers if $t=q/p$ and $p,q$ run through all pairs of integers.

\begin{ex} [1.3.1]
  Deduce that if $(a,b,c)$ is any Pythagorean triple then
  \[
    \frac{a}{c}=\frac{p^2-q^2}{p^2+q^2},\qquad\frac{b}{c}=\frac{2pq}{p^2+q^2}
  \]
  for some integers $p$ and $q$.
\end{ex}

\begin{proof}
 
\end{proof}

\begin{ex} [1.3.2]
  Use Exercise~1.3.1 to prove Euclid's formula for Pythagorean triples, assuming $b$ even. (Remember, $a$ and $b$ are not both odd.)
\end{ex}

\begin{proof}
 
\end{proof}

Some important trigonometry may be gleaned from Diophantus's method if we compare the angle at $O$ in Figure~1.4 with the angle at $Q$ in Figure~1.5. The two angles are shown in Figure~1.7, and high school geometry shows that the angle at $Q$ is half the angle at $O$.

\begin{center}
  \begin{tikzpicture}
    \draw [->] (-6,0) -- (6,0);
    \draw [->] (0,-6) -- (0,6);
    \draw (0,0) circle [radius=5];
    \draw [red] (-5,0) -- (4,3) -- (0,0);
    \draw (4,0) -- (4,3);
    \node [above left] at (-5,0) {$Q$};
    \node [below right] at (0,0) {$O$};
    \node [above right] at (5,0) {$1$};
    \node [above left] at (0,1.667) {$t$};
    \node [above right] at (1,0) {$\theta$};
    \node [above right] at (-3,0) {$\theta/2$};
    \node [right] at (6,0) {$X$};
    \node [above] at (0,6) {$Y$};
  \end{tikzpicture}
\end{center}

\begin{ex} [1.3.4]
  Why does the angle at $Q$ equal $\theta/2$? (Hint: consider angles in the red triangle.)
\end{ex}

\begin{proof}
 
\end{proof}

\begin{ex} [1.3.5]
  Use Figure~1.7 to show that $t=\tan\frac{\theta}{2}$ and
  \[
    \cos\theta=\frac{1-t^2}{1+t^2},\qquad\sin\theta=\frac{2t}{1+t^2}.
  \]
\end{ex}

\begin{proof}
 
\end{proof}

\end{document}

