\documentclass[12pt]{article}
\usepackage{amscd,amssymb,amsthm,amsxtra,exscale,latexsym,verbatim,paralist}
\usepackage{mathrsfs}
\usepackage[T1]{fontenc}
\usepackage{newtxmath,newtxtext}
\usepackage[margin=1in]{geometry}

%\usepackage{mathtools}
%\usepackage{multicol}
\usepackage{tikz}

\pagestyle{empty} 
\setlength{\parindent}{0pt} 
\setlength{\parskip}{\baselineskip}

\theoremstyle{plain}
\newtheorem{ex}{Exercise}

\renewcommand{\proofname}{Solution}

%\makeatletter
%\renewcommand*\env@matrix[1][*\c@MaxMatrixCols c]{%
%  \hskip -\arraycolsep
%  \let\@ifnextchar\new@ifnextchar
%  \array{#1}}
%\makeatother

\begin{document}

MTH 385 \qquad Homework due 2022-04-04

Like the binomial theorem, the multinomial theorem can be proved combinatorially by considering the number of ways a term $a_1^{q_1}a_2^{q_2}\cdots a_n^{q_n}$ can arise from the factors of $(a_1+a_2+\cdots+a_n)^p$.

\begin{ex} [5.9.4 rewritten]
  Prove the formula for the multinomial coefficient
  \[
    \binom{p}{q_1,q_2,\ldots,q_n}=\frac{p!}{q_1!q_2!\cdots q_n!}
  \]
  by observing that the coefficient equals the number of ways of writing a $p$-element set as a disjoint union of subsets of sizes $q_1,q_2,\ldots,q_n$.
\end{ex}

As we now know, all conic sections may be given by the following standard form equations (from Section~2.4):
\[
  \frac{x^2}{a^2}+\frac{y^2}{b^2}=1\text{ (ellipse)},\qquad y=ax^2\text{ (parabola)},\qquad  \frac{x^2}{a^2}-\frac{y^2}{b^2}=1\text{ (hyperbola)}.
\]
The reduction of an arbitrary quadratic equation in $x$ and $y$ to one of these forms depends on suitable choice of origin and axes, as Fermat and Descartes discovered. The main steps are outlined in the following exercises.

\begin{ex} [6.2.1]
  Show that a \emph{quadratic form} $ax^2+bxy+cy^2$ may be converted to a form $a'x'^2+b'y'^2$ by suitable choice of $\theta$ in the substitution
  \begin{align*}
    x &= x'\cos\theta-y'\sin\theta, \\
    y &= x'\sin\theta+y'\cos\theta,
  \end{align*}
  by checking that the coefficient of $x'y'$ is $(c-a)\sin2\theta+b\cos2\theta$.
\end{ex}

\begin{ex} [6.2.2]
  Deduce from Exercise~6.2.1 that, by suitable rotation of axes, any quadratic curve may be expressed in the form $a'x'^2+by'^2+c'x'+d'y'+e'$.
\end{ex}

\begin{ex} [6.2.3]
  If $b'=0$, but $a'\neq0$, show that the substitution $x'=x''+f$ gives either  standard-form parabola, or the ``double line'' $x''^2=0$. \\
  (Why is this called a ``double line,'' and is it a section of a cone?)
\end{ex}

\begin{ex} [6.2.4]
  If both $a'$ and $b'$ are nonzero, show that a shift of origin gives the standard form for either an ellipse or a hyperbola, or else a pair of lines.
\end{ex}

\end{document}

