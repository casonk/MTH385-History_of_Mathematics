\documentclass[12pt]{article}
\usepackage{amscd,amssymb,amsthm,amsxtra,exscale,latexsym,verbatim,paralist}
\usepackage{mathrsfs}
\usepackage[T1]{fontenc}
\usepackage{newtxmath,newtxtext}
\usepackage[margin=1in]{geometry}

%\usepackage{mathtools}
%\usepackage{multicol}
\usepackage{tikz}

\pagestyle{empty} 
\setlength{\parindent}{0pt} 
\setlength{\parskip}{\baselineskip}

\theoremstyle{plain}
\newtheorem{ex}{Exercise}

\renewcommand{\proofname}{Solution}

%\makeatletter
%\renewcommand*\env@matrix[1][*\c@MaxMatrixCols c]{%
%  \hskip -\arraycolsep
%  \let\@ifnextchar\new@ifnextchar
%  \array{#1}}
%\makeatother

\begin{document}

MTH 385 \qquad Homework due 2022-04-11

\begin{ex} [8.2.3]
  Show that the volume of the solid obtained by rotating the portion of $y=1/x$ from $x=1$ to $\infty$ about the $x$-axis is finite. Show, on the other hand, that its surface area is infinite.
\end{ex}

\begin{proof}
  Let $V$ be the volume and let $A$ be the surface area.
  \begin{align*}
    V &= \int_1^\infty\pi\left(\frac{1}{x}\right)^2\,dx \\
      &= \pi\lim_{t\to\infty}\left(\int_1^t\frac{dx}{x^2}\right) \\
      &= \pi\lim_{t\to\infty}\Bigg[-\frac{1}{x}\Bigg]_1^t  \\
      &= \pi\lim_{t\to\infty}\left[1-\frac{1}{t}\right] \\
      &= \pi \\
      \\
    A &= \int_1^\infty2\pi\frac{1}{x}\sqrt{1+\left(-\frac{1}{x^2}\right)^2}\,dx \\
      &= 2\pi\lim_{t\to\infty}\left(\int_1^t\frac{1}{x}\sqrt{1+\left(-\frac{1}{x^2}\right)^2}\,dx\right) \\
      &\geq 2\pi\lim_{t\to\infty}\left(\int_1^t\frac{dx}{x}\right) \\
      &= 2\pi\lim_{t\to\infty}\Bigg[\ln x\Bigg]_1^t \\
      &= 2\pi\lim_{t\to\infty}[\ln t-1] \\
      &= \infty
  \end{align*}
\end{proof}

Cavalieri's most elegant application of his method of indivisibles was to prove Archimedes' formula for the volume of a sphere. His argument is simpler than that of Archimedes, and it goes as follows.

\begin{ex} [8.2.4]
  Show that the slice $z=c$ of the sphere $x^2+y^2+z^2=1$ has the same area as the slice $z=c$ of the cylinder $  x^2+y^2=1$ outside the cone $x^2+y^2=z^2$ (Figure~8.2).
\end{ex}

\begin{proof}
  Out of laziness I have chosen not to replicate Figure~8.2.
  
  Let's compute the area of the slice $z=c$ of the sphere $x^2+y^2+z^2=1$. It is the circle $x^2+y^2=1-c^2$ of radius $\sqrt{1-c^2}$ and area $\pi(1-c^2)$. The slice $z=c$ of the cylinder $x^2+y^2=1$ is the circle $x^2+y^2=1$ of radius $1$ and area $\pi$.  The slice $z=c$ of the cone $x^2+y^2=1$ is the circle $x^2+y^2=c^2$ of radius $c$ and area $\pi c^2$. Evidently, $\pi(1-c^2)=\pi-\pi c^2$. 
\end{proof}

\begin{ex} [8.2.5]
  Deduce from Exercise~8.2.4, and the known volume of the cone, that the volume of the sphere is $2/3$ the volume of the circumscribing cylinder.
\end{ex}

\begin{proof}
  We know the volume of a cone of radius $r$ and height $h$ is $\frac{1}{3}\pi r^2h$. This is $1/3$ the volume of the enclosing cylinder. Thus, the volume of the sphere is $2/3$ that of the circumscribing cylinder.
\end{proof}

The examples in Exercise~8.3.1 and Exercise~8.3.2 show how tangents can be found by looking for double roots, though it requires some foresight to make the right substitution. With calculus, the process is more mechanical.

\begin{ex} [8.3.3]
  Derive the formula of Hudde and Sluse by differentiating $\sum a_{ij}x^iy^j=0$ with respect to $x$.
\end{ex}

\begin{proof}
  \begin{align*}
    \frac{d}{dx}\left(\sum a_{ij}x^iy^j\right)                              &= 0 \\
    \sum ia_{ij}x^{i-1}y^j+\sum ja_{ij}x^iy^{j-1}\frac{dy}{dx}              &= 0 \\
    \sum ia_{ij}x^{i-1}y^j+\left(\sum ja_{ij}x^iy^{j-1}\right)\frac{dy}{dx} &= 0 \\
    \left(\sum ja_{ij}x^iy^{j-1}\right)\frac{dy}{dx}                        &= -\sum ia_{ij}x^{i-1}y^j \\
    \frac{dy}{dx}                                                           &= -\frac{\sum ia_{ij}x^{i-1}y^j}{\sum ja_{ij}x^iy^{j-1}}
  \end{align*}
\end{proof}

\begin{ex} [8.3.4]
  Use differentiation to find the tangent to the folium $x^3+y^3=3axy$ at the point $(b,c)$.
\end{ex}

\begin{proof}
  First, determine the slope of the tangent line.
  \begin{align*}
    \frac{d}{dx}(x^3+y^3)   &= \frac{d}{dx}(3axy) \\
    3x^2+3y^2\frac{dy}{dx}  &= 3ay+3ax\frac{dy}{dx} \\
    x^2+y^2\frac{dy}{dx}    &= ay+ax\frac{dy}{dx} \\
    (y^2-ax)\frac{dy}{dx}   &= ay-x^2 \\
    \frac{dy}{dx}           &= \frac{ay-x^2}{y^2-ax} \\
    \frac{dy}{dx}           &= \frac{ac-b^2}{c^2-ab}
  \end{align*}
  Then, use point-slope form.
  \[
    y=c+\frac{ac-b^2}{c^2-ab}(x-b)
  \]
\end{proof}

\end{document}

