\documentclass[12pt]{article}
\usepackage{amscd,amssymb,amsthm,amsxtra,exscale,latexsym,verbatim,paralist}
\usepackage{mathrsfs}
\usepackage[T1]{fontenc}
\usepackage{newtxmath,newtxtext}
\usepackage[margin=1in]{geometry}

%\usepackage{mathtools}
%\usepackage{multicol}
\usepackage{tikz}

\pagestyle{empty} 
\setlength{\parindent}{0pt} 
\setlength{\parskip}{\baselineskip}

\theoremstyle{plain}
\newtheorem{ex}{Exercise}

\renewcommand{\proofname}{Solution}

%\makeatletter
%\renewcommand*\env@matrix[1][*\c@MaxMatrixCols c]{%
%  \hskip -\arraycolsep
%  \let\@ifnextchar\new@ifnextchar
%  \array{#1}}
%\makeatother

\begin{document}

MTH 385 \qquad Homework due 2022-04-11

\begin{ex} [8.2.3]
  Show that the volume of the solid obtained by rotating the portion of $y=1/x$ from $x=1$ to $\infty$ about the $x$-axis is finite. Show, on the other hand, that its surface area is infinite.
\end{ex}


Cavalieri's most elegant application of his method of indivisibles was to prove Archimedes' formula for the volume of a sphere. His argument is simpler than that of Archimedes, and it goes as follows.

\begin{ex} [8.2.4]
  Show that the slice $z=c$ of the sphere $x^2+y^2+z^2=1$ has the same area as the slice $z=c$ of the cylinder $  x^2+y^2=1$ outside the cone $x^2+y^2=z^2$ (Figure~8.2).
\end{ex}

\begin{ex} [8.2.5]
  Deduce from Exercise~8.2.4, and the known volume of the cone, that the volume of the sphere is $2/3$ the volume of the circumscribing cylinder.
\end{ex}

The examples in Exercise~8.3.1 and Exercise~8.3.2 show how tangents can be found by looking for double roots, though it requires some foresight to make the right substitution. With calculus, the process is more mechanical.

\begin{ex} [8.3.3]
  Derive the formula of Hudde and Sluse by differentiating $\sum a_{ij}x^iy^j=0$ with respect to $x$.
\end{ex}

\begin{ex} [8.3.4]
  Use differentiation to find the tangent to the folium $x^3+y^3=3axy$ at the point $(b,c)$.
\end{ex}

\end{document}

