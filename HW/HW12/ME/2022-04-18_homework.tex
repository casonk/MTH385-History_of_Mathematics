\documentclass[12pt]{article}
\usepackage{amscd,amssymb,amsthm,amsxtra,exscale,latexsym,verbatim,paralist}
\usepackage{mathrsfs}
\usepackage[T1]{fontenc}
\usepackage{newtxmath,newtxtext}
\usepackage[left = 2cm, top = 2cm, bottom = 2cm, right = 2cm]{geometry}

\usepackage{hyperref}
\usepackage{tikz}
\usetikzlibrary{patterns}

\newcommand{\XB}{\color{black}}
\newcommand{\XBB}{\color{blue}}
\newcommand{\XV}{\color{violet}}
\newcommand{\XR}{\color{red}}
\newcommand{\ds}{\displaystyle}

\setlength{\parindent}{0pt} 
\setlength{\parskip}{\baselineskip}

\theoremstyle{plain}
\newtheorem{ex}{Exercise}

\renewcommand{\proofname}{Solution}

\begin{document}

\title{\textbf{MTH385}: History of Mathematics - Homework \#12}
\date{\today}
\author{\XV\textit{\large{\href{https://github.com/casonk}{Cason Konzer}}}\XB}

\maketitle

\hrulefill

\newpage

The equation relating the series for $\frac{\pi}{4}$ to the continued fraction for $\frac{4}{\pi}$, namely
\[
  1 - \frac{1}{3} + \frac{1}{5} - \frac{1}{7} + \cdots = \cfrac{1}{1 + \cfrac{1^{2}}{2 + \cfrac{3^{2}}{2 + \cfrac{5^{2}}{2 + \cfrac{7^{2}}{2 + \cdots}}}}}
\]
follows immediately from a more general equation
\[
  \frac{1}{A} - \frac{1}{B} + \frac{1}{C} - \frac{1}{D} + \cdots = \cfrac{1}{A + \cfrac{A^{2}}{B - A + \cfrac{B^{2}}{C - B + \cfrac{C^{2}}{D - C + \cdots}}}}
\]
proved by Euler~(1748a),~p.~311. The following exercises give a proof of Euler's result.

%%%%%%%%%%%%%%%%%%%%%%%%%%%%%%%%%%%%%%%%%%%%%%%%%%%%%%%%%%%%%%%%%%%%%%%%%%%%%%%%%%%%%%%%%%%%%%%%%%%%%%%%%%%%%
%%%%%%%%%%%%%%%%%%     #1     %%%%%%%%%%%%%%%%%%%%%%%%%%%%%%%%%%%%%%%%%%%%%%%%%%%%%%%%%%%%%%%%%%%%%%%%%%%%%%%
%%%%%%%%%%%%%%%%%%%%%%%%%%%%%%%%%%%%%%%%%%%%%%%%%%%%%%%%%%%%%%%%%%%%%%%%%%%%%%%%%%%%%%%%%%%%%%%%%%%%%%%%%%%%%

\XBB\hrulefill\XB \\
\begin{ex} [8.4.3]
  Check that
  \[
    \frac{1}{A} - \frac{1}{B} = \cfrac{1}{A + \cfrac{A^{2}}{B - A}}
  \]
\end{ex}
\XBB\hrulefill\XB \\

\begin{proof}
  \ \\

  \begin{itemize}
    \item $ \ds \frac{1}{A} - \frac{1}{B} = \frac{B - A}{AB} = \frac{1}{\cfrac{AB}{B -A}} = \frac{1}{\cfrac{AB - A^{2} + A^{2}}{B -A}} = \frac{1}{\cfrac{A(B - A) + A^{2}}{B -A}} = \cfrac{1}{A + \cfrac{A^{2}}{B - A}} $.
  \end{itemize}

\end{proof}

\newpage
%%%%%%%%%%%%%%%%%%%%%%%%%%%%%%%%%%%%%%%%%%%%%%%%%%%%%%%%%%%%%%%%%%%%%%%%%%%%%%%%%%%%%%%%%%%%%%%%%%%%%%%%%%%%%
%%%%%%%%%%%%%%%%%%     #2     %%%%%%%%%%%%%%%%%%%%%%%%%%%%%%%%%%%%%%%%%%%%%%%%%%%%%%%%%%%%%%%%%%%%%%%%%%%%%%%
%%%%%%%%%%%%%%%%%%%%%%%%%%%%%%%%%%%%%%%%%%%%%%%%%%%%%%%%%%%%%%%%%%%%%%%%%%%%%%%%%%%%%%%%%%%%%%%%%%%%%%%%%%%%%

\XBB\hrulefill\XB \\
\begin{ex} [8.4.4]
  When $ \frac{1}{B} $ on the left side in Exercise~8.4.3 is replaced by $ \frac{1}{B} - \frac{1}{C} $, 
  which equals $ \frac{1}{B - \frac{B^{2}}{C - B}} $ by Exercise~8.4.3, show that $ B $ on the right should be replaced by $ B + \frac{B^{2}}{C - B} $. Hence show that
  \[
    \frac{1}{A} - \frac{1}{B} + \frac{1}{C} = \cfrac{1}{A + \cfrac{A^{2}}{B - A + \cfrac{B^{2}}{C - B}}}
  \]
\end{ex}
\XBB\hrulefill\XB \\

\begin{proof}
  \ \\

  \begin{itemize}
    \item $ \ds Let \quad B - \frac{B^{2}}{C - B} = X $.
    \item $ \ds \frac{1}{A} - \frac{1}{B} + \frac{1}{C} = \frac{1}{A} - \Bigl( \frac{1}{B} - \frac{1}{C} \Bigr) = \frac{1}{A} - \frac{1}{B - \cfrac{B^{2}}{C - B}} = \frac{1}{A} - \frac{1}{X} = \cfrac{1}{A + \cfrac{A^{2}}{X - A}} = \cfrac{1}{A + \cfrac{A^{2}}{B - A + \cfrac{B^{2}}{C - B}}} $.
  \end{itemize}

\end{proof}

\newpage

Thus when we modify the tail end of the series (replacing $ \frac{1}{B} $ by by $ \frac{1}{B} - \frac{1}{C} $), only the tail end of the continued fraction is affected. This situation continues:

%%%%%%%%%%%%%%%%%%%%%%%%%%%%%%%%%%%%%%%%%%%%%%%%%%%%%%%%%%%%%%%%%%%%%%%%%%%%%%%%%%%%%%%%%%%%%%%%%%%%%%%%%%%%%
%%%%%%%%%%%%%%%%%%     #3     %%%%%%%%%%%%%%%%%%%%%%%%%%%%%%%%%%%%%%%%%%%%%%%%%%%%%%%%%%%%%%%%%%%%%%%%%%%%%%%
%%%%%%%%%%%%%%%%%%%%%%%%%%%%%%%%%%%%%%%%%%%%%%%%%%%%%%%%%%%%%%%%%%%%%%%%%%%%%%%%%%%%%%%%%%%%%%%%%%%%%%%%%%%%%

\XBB\hrulefill\XB \\
\begin{ex} [8.4.5]
  Generalize your argument in Exercise~8.4.4 to obtain a continued fraction for a series with $n$ terms, and hence prove Euler's equation.
\end{ex}
\XBB\hrulefill\XB \\

\begin{proof}
  \ \\

  Consider \dots

  \begin{itemize}
    \item $ \ds \frac{1}{A_{0}} - \frac{1}{A_{1}} + \frac{1}{A_{2}} - \frac{1}{A_{3}} +- \cdots +- \frac{1}{A_{n-3}} + \frac{1}{A_{n-2}} - \frac{1}{A_{n-1}} + \frac{1}{A_{n}} $.
    \item $ \ds = \frac{1}{A_{0}} - \Bigl( \frac{1}{A_{1}} - \frac{1}{A_{2}} + \frac{1}{A_{3}} -+ \cdots -+ \frac{1}{A_{n-3}} - \frac{1}{A_{n-2}} + \frac{1}{A_{n-1}} - \frac{1}{A_{n}} \Bigr) $.
    \item $ \ds = \cdots $.
    \item $ \ds = \frac{1}{A_{0}} - \Bigl( \frac{1}{A_{1}} - \Bigl( \frac{1}{A_{2}} - \Bigl( \frac{1}{A_{3}} - \Bigl( + \cdots +- \Bigl( \frac{1}{A_{n-3}} - \Bigl( \frac{1}{A_{n-2}} - \frac{1}{A_{n-1}} + \frac{1}{A_{n}} \Bigr) \Bigr) \Bigr) \Bigr) \Bigr) \Bigr) $.
    \item $ \ds = \frac{1}{A_{0}} - \Bigl( \frac{1}{A_{1}} - \Bigl( \frac{1}{A_{2}} - \Bigl( \frac{1}{A_{3}} - \Bigl( + \cdots +- \Bigl( \frac{1}{A_{n-3}} - \Bigl( \frac{1}{A_{n-2}} - \Bigl(  \frac{1}{A_{n-1}} - \frac{1}{A_{n}} \Bigr) \Bigr) \Bigr) \Bigr) \Bigr) \Bigr) \Bigr) $.
  \end{itemize}

  Now making similar substitutions \dots

  \begin{itemize}
    \item $ \ds = \frac{1}{A_{0}} - \Bigl( \frac{1}{A_{1}} - \Bigl( \frac{1}{A_{2}} - \Bigl( \frac{1}{A_{3}} - \Bigl( + \cdots +- \Bigl( \frac{1}{A_{n-3}} - \Bigl( \frac{1}{A_{n-2}} - \frac{1}{X_{n}} \Bigr) \Bigr) \Bigr) \Bigr) \Bigr) \Bigr) $.
    \item $ \ds = \frac{1}{A_{0}} - \Bigl( \frac{1}{A_{1}} - \Bigl( \frac{1}{A_{2}} - \Bigl( \frac{1}{A_{3}} - \Bigl( + \cdots +- \Bigl( \frac{1}{A_{n-3}} -  \frac{1}{X_{n-1}} \Bigr) \Bigr) \Bigr) \Bigr) \Bigr) $.
    \item $ \ds = \cdots $.
    \item $ \ds = \frac{1}{A_{0}} - \Bigl( \frac{1}{A_{1}} - \frac{1}{X_{2}} \Bigr) $.
    \item $ \ds = \frac{1}{A_{0}} - \frac{1}{X_{1}} $.
  \end{itemize}

  \newpage

  Which by the same method arrives at Euler's equation \dots

  \begin{itemize}
    \item $ \ds \frac{1}{A_{0}} - \frac{1}{X_{1}} = \cfrac{1}{A_{0} + \cfrac{A_{0}^{2}}{X_{1} - A_{0}}} $.
    \item $ \ds = \frac{1}{A_{0}} - \Bigl( \frac{1}{A_{1}} - \frac{1}{X_{2}} \Bigr) = \cfrac{1}{A_{0} + \cfrac{A_{0}^{2}}{A_{1} - A_{0} + \cfrac{A_{1}^{2}}{X_{2} - A_{1}}}} $.
    \item $ \ds = \frac{1}{A_{0}} - \Bigl( \frac{1}{A_{1}} - \Bigl( \frac{1}{A_{2}} -  \frac{1}{X_{3}} \Bigr) \Bigr) = \cfrac{1}{A_{0} + \cfrac{A_{0}^{2}}{A_{1} - A_{0} + \cfrac{A_{1}^{2}}{A_{2} - A_{1} + \cfrac{A_{2}^{2}}{X_{3} - A_{2}}}}} $.
    \item $ \ds = \cdots $.
    \item $ \ds = \frac{1}{A_{0}} - \Bigl( \frac{1}{A_{1}} - \Bigl( \frac{1}{A_{2}} - \Bigl( \frac{1}{A_{3}} - \Bigl( + \cdots +- \Bigl( \frac{1}{A_{n-3}} - \Bigl( \frac{1}{A_{n-2}} - \Bigl(  \frac{1}{A_{n-1}} - \frac{1}{A_{n}} \Bigr) \Bigr) \Bigr) \Bigr) \Bigr) \Bigr) \Bigr) $.
    \item $ \ds = \cfrac{1}{A_{0} + \cfrac{A_{0}^{2}}{A_{1} - A_{0} + \cfrac{A_{1}^{2}}{A_{2} - A_{1} + \cfrac{A_{2}^{2}}{A_{3} - A_{2} + \cfrac{A_{\dots}^{2}}{A_{\dots} - A_{n - 3} + \cfrac{A_{n - 2}^{2}}{A_{n - 1} - A_{n - 2} + \cfrac{A_{n - 1}^{2}}{A_{n} - A_{n - 1}}}}}}}} $.
  \end{itemize}

\end{proof}

\newpage

Exercise~8.5.2 shows why the inverse function $ x = e^{y} - 1 $ has a power series that begins
\[
  y + \frac{1}{2}y^{2} + \frac{1}{6}y^{3} + \cdots.
\]

%%%%%%%%%%%%%%%%%%%%%%%%%%%%%%%%%%%%%%%%%%%%%%%%%%%%%%%%%%%%%%%%%%%%%%%%%%%%%%%%%%%%%%%%%%%%%%%%%%%%%%%%%%%%%
%%%%%%%%%%%%%%%%%%     #4     %%%%%%%%%%%%%%%%%%%%%%%%%%%%%%%%%%%%%%%%%%%%%%%%%%%%%%%%%%%%%%%%%%%%%%%%%%%%%%%
%%%%%%%%%%%%%%%%%%%%%%%%%%%%%%%%%%%%%%%%%%%%%%%%%%%%%%%%%%%%%%%%%%%%%%%%%%%%%%%%%%%%%%%%%%%%%%%%%%%%%%%%%%%%%

\XBB\hrulefill\XB \\
\begin{ex} [8.5.3]
  Show that the binomial series gives
  \[
    \frac{1}{\sqrt{1 - t^{2}}} = 1 + \frac{1}{2}t^{2} + \frac{1 \cdot 3}{2 \cdot 4}t^{4} + \frac{1 \cdot 3 \cdot 5}{2 \cdot 4 \cdot 6}t^{6} + \cdots.
  \]
\end{ex}
\XBB\hrulefill\XB \\

\begin{proof}
  \ \\

  \begin{itemize}
    \item $ \ds (1 + a)^{p} = 1 + pa + \frac{p(p - 1)}{2}a^{2} + \frac{p(p - 1)(p - 2)}{6}a^{3} + \cdots $.
    \item $ \ds \frac{1}{\sqrt{1 - t^{2}}} = (1 + (-t^{2}))^{-1/2} = 1 + -\frac{1}{2} (-t^{2}) + \frac{1}{2!} \Bigl( -\frac{1}{2} \Bigr) \Bigl( -\frac{3}{2} \Bigr) (-t^{2})^{2} + \frac{1}{3!} \Bigl( -\frac{1}{2} \Bigr) \Bigl( -\frac{3}{2} \Bigr) \Bigl( -\frac{5}{2} \Bigr) (-t^{2})^{3} + \cdots $.
    \item $ \ds = 1 + \frac{1}{2}t^{2} + \frac{1 \cdot 3}{2 \cdot 4}t^{4} + \frac{1 \cdot 3 \cdot 5}{2 \cdot 4 \cdot 6}t^{6} + \cdots $.
  \end{itemize}

\end{proof}

\newpage

%%%%%%%%%%%%%%%%%%%%%%%%%%%%%%%%%%%%%%%%%%%%%%%%%%%%%%%%%%%%%%%%%%%%%%%%%%%%%%%%%%%%%%%%%%%%%%%%%%%%%%%%%%%%%
%%%%%%%%%%%%%%%%%%     #5     %%%%%%%%%%%%%%%%%%%%%%%%%%%%%%%%%%%%%%%%%%%%%%%%%%%%%%%%%%%%%%%%%%%%%%%%%%%%%%%
%%%%%%%%%%%%%%%%%%%%%%%%%%%%%%%%%%%%%%%%%%%%%%%%%%%%%%%%%%%%%%%%%%%%%%%%%%%%%%%%%%%%%%%%%%%%%%%%%%%%%%%%%%%%%

\XBB\hrulefill\XB \\
\begin{ex} [8.5.4]
  Use Exercise~8.5.3 and $ \sin^{-1}(x) = \int_{0}^{x} dt/ \sqrt{1 - t^{2}} $ to derive Newton's series for $ \sin^{-1}(x) $.
\end{ex}
\XBB\hrulefill\XB \\

\begin{proof}
  \ \\

  \begin{itemize}
    \item $ \ds \sin^{-1}(x) = \int_{0}^{x} 1 + \frac{1}{2}t^{2} + \frac{1 \cdot 3}{2 \cdot 4}t^{4} + \frac{1 \cdot 3 \cdot 5}{2 \cdot 4 \cdot 6}t^{6} + \cdots \,dt $.
    \item $ \ds = \int_{0}^{x} \,dt + \int_{0}^{x} \frac{1}{2}t^{2} \,dt + \int_{0}^{x} \frac{3}{8}t^{4} \,dt + \int_{0}^{x} \frac{5}{16}t^{6} \,dt + \int_{0}^{x} \cdots \,dt $.
    \item $ \ds = t \Big|_{0}^{x} + \frac{1}{6}t^{3} \Big|_{0}^{x} + \frac{3}{40}t^{5} \Big|_{0}^{x} + \frac{5}{112}t^{7} \Big|_{0}^{x} + \cdots $.
    \item $ \ds = x + \frac{1}{6}x^{3} + \frac{3}{40}x^{5} + \frac{5}{112}x^{7} + \cdots $.
  \end{itemize}

\end{proof}

\newpage

\end{document}

