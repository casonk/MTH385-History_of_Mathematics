\documentclass[12pt]{article}
\usepackage{amscd,amssymb,amsthm,amsxtra,exscale,latexsym,verbatim,paralist}
\usepackage{mathrsfs}
\usepackage[T1]{fontenc}
\usepackage{newtxmath,newtxtext}
\usepackage[margin=1in]{geometry}

%\usepackage{mathtools}
%\usepackage{multicol}
\usepackage{tikz}

\pagestyle{empty} 
\setlength{\parindent}{0pt} 
\setlength{\parskip}{\baselineskip}

\theoremstyle{plain}
\newtheorem{ex}{Exercise}

\renewcommand{\proofname}{Solution}

%\makeatletter
%\renewcommand*\env@matrix[1][*\c@MaxMatrixCols c]{%
%  \hskip -\arraycolsep
%  \let\@ifnextchar\new@ifnextchar
%  \array{#1}}
%\makeatother

\begin{document}

MTH 385 \qquad Homework due 2022-04-18

The equation relating the series for $\frac{\pi}{4}$ to the continued fraction for $\frac{4}{\pi}$, namely
\[
  1-\frac{1}{3}+\frac{1}{5}-\frac{1}{7}+\cdots=\cfrac{1}{1+\cfrac{1^2}{2+\cfrac{3^2}{2+\cfrac{5^2}{2+\cfrac{7^2}{2+\cdots}}}}}
\]
follows immediately from a more general equation
\[
  \frac{1}{A}-\frac{1}{B}+\frac{1}{C}-\frac{1}{D}+\cdots=\cfrac{1}{A+\cfrac{A^2}{B-A+\cfrac{B^2}{C-B+\cfrac{C^2}{D-C+\cdots}}}}
\]
proved by Euler~(1748a),~p.~311. The following exercises give a proof of Euler's result.

\begin{ex} [8.4.3]
  Check that
  \[
    \frac{1}{A}-\frac{1}{B}=\cfrac{1}{A+\cfrac{A^2}{B-A}}
  \]
\end{ex}

\begin{ex} [8.4.4]
  When $\frac{1}{B}$ on the left side in Exercise~8.4.3 is replaced by $\frac{1}{B}-\frac{1}{C}$, which equals $\frac{1}{B-\frac{B^2}{C-B}}$ by Exercise~8.4.3, show that $B$ on the right should be replaced by $B+\frac{B^2}{C-B}$. Hence show that
  \[
    \frac{1}{A}-\frac{1}{B}+\frac{1}{C}-\frac{1}{D}+\cdots=\cfrac{1}{A+\cfrac{A^2}{B-A+\cfrac{B^2}{C-B}}}
  \]
\end{ex}

Thus when we modify the tail end of the series (replacing $\frac{1}{B}$ by by $\frac{1}{B}-\frac{1}{C}$), only the tail end of the continued fraction is affected. This situation continues:

\begin{ex} [8.4.5]
  Generalize your argument in Exercise~8.4.4 to obtain a continued fraction for a series with $n$ terms, and hence prove Euler's equation.
\end{ex}

Exercise~8.5.2 shows why the inverse function $x=e^y-1$ has a power series that begins
\[
  y+\frac{1}{2}y^2+\frac{1}{6}y^3+\cdots.
\]

\begin{ex} [8.5.3]
  Show that the binomial series gives
  \[
    \frac{1}{\sqrt{1-t^2}}=1+\frac{1}{2}t^2+\frac{1\cdot3}{2\cdot4}t^4+\frac{1\cdot3\cdot5}{2\cdot4\cdot6}t^6+\cdots.
  \]
\end{ex}

\begin{ex} [8.5.4]
  Use Exercise~8.5.3 and $\sin^{-1}x=\int_0^xdt/\sqrt{1-t^2}$ to derive Newton's series for $\sin^{-1}x$.
\end{ex}

\end{document}

