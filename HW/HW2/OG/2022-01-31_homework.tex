\documentclass[12pt]{article}
\usepackage{amscd,amssymb,amsthm,amsxtra,exscale,latexsym,verbatim,paralist}
\usepackage{mathrsfs}
\usepackage[T1]{fontenc}
\usepackage{newtxmath,newtxtext}
\usepackage[margin=1in]{geometry}

%\usepackage{mathtools}
%\usepackage{multicol}
\usepackage{tikz}

\pagestyle{empty} 
\setlength{\parindent}{0pt} 
\setlength{\parskip}{\baselineskip}

\theoremstyle{plain}
\newtheorem{ex}{Exercise}

\renewcommand{\proofname}{Solution}

%\makeatletter
%\renewcommand*\env@matrix[1][*\c@MaxMatrixCols c]{%
%  \hskip -\arraycolsep
%  \let\@ifnextchar\new@ifnextchar
%  \array{#1}}
%\makeatother

\begin{document}

MTH 385 \qquad Homework due 2022-01-31


\begin{ex} [1.5.2]
  Show that the square of $2q+1$ is in fact of the form $4s+1$, and hence explain why every integer square leaves remainder $0$ or $1$ on division by $4$.
\end{ex}

\begin{proof}
 
\end{proof}

\begin{ex} [2.1.1]
  Explain how Common Notions~1 and 4 may be interpreted as the transitive and reflexive properties. Note that the natural way to write Common Notion 1 symbolically is slightly different from the statement of transitivity above.
\end{ex}

\begin{proof}
 
\end{proof}

\begin{ex} [2.1.2]
  Show that the symmetric property follows from Euclid's Common Notions~1 and 4.
\end{ex}

\begin{ex} [2.2.1]
  Show that $\frac{\text{circumradius}}{\text{inradius}}=\sqrt{3}$ for both the cube and the octahedron.
\end{ex}

\begin{proof}
 
\end{proof}

\begin{ex} [2.2.2]
  Check Pacioli's construction: use the Pythagorean theorem to show that $AB=BC=CA$ in Figure~2.2. (It may help to use the additional fact that $\tau=(1+\sqrt{5})/2$ satisfies $\tau^2=\tau+1$.)
\end{ex}

\begin{proof}
 
\end{proof}

\end{document}

