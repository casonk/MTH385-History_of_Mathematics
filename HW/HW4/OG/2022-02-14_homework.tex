\documentclass[12pt]{article}
\usepackage{amscd,amssymb,amsthm,amsxtra,exscale,latexsym,verbatim,paralist}
\usepackage{mathrsfs}
\usepackage[T1]{fontenc}
\usepackage{newtxmath,newtxtext}
\usepackage[margin=1in]{geometry}

%\usepackage{mathtools}
%\usepackage{multicol}
\usepackage{tikz}

\pagestyle{empty} 
\setlength{\parindent}{0pt} 
\setlength{\parskip}{\baselineskip}

\theoremstyle{plain}
\newtheorem{ex}{Exercise}

\renewcommand{\proofname}{Solution}

%\makeatletter
%\renewcommand*\env@matrix[1][*\c@MaxMatrixCols c]{%
%  \hskip -\arraycolsep
%  \let\@ifnextchar\new@ifnextchar
%  \array{#1}}
%\makeatother

\begin{document}

MTH 385 \qquad Homework due 2022-02-14

The simplest epicyclic curve is the \emph{cardioid} (``heart-shape''), which results from a circle rolling on a fixed circle of the same size.

\begin{ex} [2.5.4]
  Show that if both circles have radius $1$, and we follow the point on the rolling circle initially at $(1,0)$, then the cardioid it traces out has parametric equations
  \begin{align*}
    x &= 2\cos\theta-\cos2\theta, \\
    y &= 2\sin\theta-\sin2\theta.
  \end{align*}
\end{ex}

The cardioid is an algebraic curve. Its cartesian equation may be hard to dis- cover, but it is easy to verify, especially if one has a computer algebra system.

\begin{ex} [2.5.5]
  Check that the point $(x,y)$ on the cardioid satisfies
  \[
    (x^2+y^2-1)^2=4((x-1)^2+y^2).
  \]
\end{ex}

\begin{ex} [3.2.3]
  Show that the $k$th pentagonal number is $(3k^2-k)/2$.
\end{ex}

\begin{ex} [3.2.4]
  Show that each square is the sum of two consecutive triangular numbers.
\end{ex}

Euclid's theorem about perfect numbers depends on the prime divisor property, which will be proved in the next section. Assuming this for the moment, it follows that if $2^n-1$ is a prime $p$, then the proper divisors of $2^{n-1}p$ (those unequal to $2^{n-1}p$ itself) are
\[
  1,2,2^2,\ldots,2^{n-1}\text{ and }p,2p,2^2p,\ldots,2^{n-2}p.
\]

\begin{ex} [3.2.5]
  Given that the divisors of $2^{n-1}p$ are those just listed, show that $2^{n-1}p$ is perfect when $p=2^n-1$ is prime.
\end{ex}

\end{document}

