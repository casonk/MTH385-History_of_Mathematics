\documentclass[12pt]{article}
\usepackage{amscd,amssymb,amsthm,amsxtra,exscale,latexsym,verbatim,paralist}
\usepackage{mathrsfs}
\usepackage[T1]{fontenc}
\usepackage{newtxmath,newtxtext}
\usepackage[margin=1in]{geometry}

%\usepackage{mathtools}
%\usepackage{multicol}
\usepackage{tikz}

\pagestyle{empty} 
\setlength{\parindent}{0pt} 
\setlength{\parskip}{\baselineskip}

\theoremstyle{plain}
\newtheorem{ex}{Exercise}

\renewcommand{\proofname}{Solution}

%\makeatletter
%\renewcommand*\env@matrix[1][*\c@MaxMatrixCols c]{%
%  \hskip -\arraycolsep
%  \let\@ifnextchar\new@ifnextchar
%  \array{#1}}
%\makeatother

\begin{document}

MTH 385 \qquad Homework due 2022-02-21

\begin{ex} [3.3.2]
  Show that, for any integers $a$ and $b$, there are integers $m$ and $n$ such that $\gcd(a,b)=ma+nb$.
\end{ex}

This in turn gives a general way to find integer solutions of linear equations.

\begin{ex} [3.3.3]
  Deduce from Exercise~3.3.2 that the equation $ax+by=c$ with integer coefficients $a$, $b$, and $c$ has an integer solution $x$, $y$ if $\gcd(a,b)$ divides $c$.
\end{ex}

\begin{ex} [3.3.5]
  (Solution of linear Diophantine equations) Give a test to decide, for any given integers $a$, $b$, $c$, whether there are integers $x$, $y$ such that
  \[
    ax+by=c.
  \]
\end{ex}

\begin{ex} [3.4.3]
  Show that
  \[
    \sqrt{2}=1+\cfrac{1}{2+\cfrac{1}{2+\cfrac{1}{2+\cfrac{1}{\ddots}}}}
  \]
\end{ex}

Exercise~3.4.3 implies that $\sqrt{2}+1$ is the periodic continued fraction
\[
  2+\cfrac{1}{2+\cfrac{1}{2+\cfrac{1}{2+\cfrac{1}{\ddots}}}}
\]

\begin{ex} [3.4.4]
  Show that $\sqrt{3}+1$ also has a periodic continued fraction, and hence derive the continued fraction for $\sqrt{3}$.
\end{ex}

\end{document}

