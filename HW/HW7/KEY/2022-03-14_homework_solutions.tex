\documentclass[12pt]{article}
\usepackage{amscd,amssymb,amsthm,amsxtra,exscale,latexsym,verbatim,paralist}
\usepackage{mathrsfs}
\usepackage[T1]{fontenc}
\usepackage{newtxmath,newtxtext}
\usepackage[margin=1in]{geometry}

%\usepackage{mathtools}
%\usepackage{multicol}
\usepackage{tikz}

\pagestyle{empty} 
\setlength{\parindent}{0pt} 
\setlength{\parskip}{\baselineskip}

\theoremstyle{plain}
\newtheorem{ex}{Exercise}

\renewcommand{\proofname}{Solution}

%\makeatletter
%\renewcommand*\env@matrix[1][*\c@MaxMatrixCols c]{%
%  \hskip -\arraycolsep
%  \let\@ifnextchar\new@ifnextchar
%  \array{#1}}
%\makeatother

\begin{document}

MTH 385 \qquad Homework due 2022-03-14 Solutions

Consider the intersection of two circles. Fortunately, it is easy to reduce these two quadratic equations to the case handled in Exercise~5.3.4.

\begin{ex} [5.3.5]
  The equations of any two circles can be written in the form
  \begin{align*}
    (x-a)^2+(y-b)^2 &= r^2 \\
    (x-c)^2+(y-d)^2 &= s^2
  \end{align*}
  Explain why. Now subtract one of these equations from the other, and hence show that their common solutions can be found by rational operations and square roots.
\end{ex}

\begin{proof}
  Notice that the locus of points at a distance $r$ from the point $(a,b)$ is
  \[
    \left\{(x,y)\in\mathbb{R}^2\mid\sqrt{(x-a)^2+(y-b)^2}=r\right\}.
  \]
  That is, $(x-a)^2+(y-b)^2=r^2$ is the equation of of the circle of radius $r$ centered at $(a,b)$.

  Notice that the circles are disjoint or the same if $(a,b)=(c,d)$.

  We will find a line whose intersection with either circle is the same as the intersection of the two circles. Then we will appeal to the result of Exercise~5.3.4. Subtract the second equation from the first.
  \begin{align*}
    (x-a)^2-(x-c)^2+(y-b)^2-(y-d)^2                     &= r^2-s^2 \\
    x^2-2ax+a^2-(x^2-2cx+c^2)+y^2-2by+b^2-(y^2-2dy+d^2) &= r^2-s^2 \\
    2(c-a)x+2(d-b)y                                     &= r^2-s^2-a^2+c^2-b^2-d^2
  \end{align*}
  Notice that this last equation is a general equation of a line if $(a,b)\neq(c,d)$. If some point $(x_0,y_0)$ is on both circles, then $x=x_0,y=y_0$ is a solution to the equations of both of the circles. Thus, $x=x_0,y=y_0$ is also a solution to the equation for the line. So, if $(x_0,y_0)$ is on both circles, then $(x_0,y_0)$ is also on the line. Moreover, since the first of the circle equations is (a rearrangement of) the sum of the second circle equation and the equation of the line, any point on the line and the second circle is also on the first circle. Hence, the intersection of the second circle and the line is the same set as the intersection of the two circles.

  Similarly, the intersection of the first circle and the line is the same as the intersection of the two circles since the second circle equation is (a rearrangement of) the first circle equation minus the equation of the line. 
\end{proof}

When a sequence of quadratic equations is solved, the solution may involve \emph{nested} square roots, such as $\sqrt{(5+\sqrt{5})/2}$. This very number, in fact, occurs in the icosahedron, as one sees from Pacioli's construction in Section~2.2.

\begin{ex} [5.3.6]
  Show that the diagonal of a golden rectangle (which is also the diameter of an icosahedron of edge length $1$) is $\sqrt{(5+\sqrt{5})/2}$.
\end{ex}

\begin{proof}
  Recall: $(1+\sqrt{5})/2$ is the length of the longer edges of the golden rectangle whose shorter edges have length $1$. Now, use the Pythogarean Theorem.
  \[
    \sqrt{1^2+\left(\frac{1+\sqrt{5}}{2}\right)^2}=\sqrt{\frac{4+1+2\sqrt{5}+5}{4}}=\sqrt{\frac{5+\sqrt{5}}{2}}
  \]
\end{proof}

We know from Exercise~5.4.1 that $\sqrt[3]{2}$ is not in $F_0$, but if it is constructible it will occur in some $F_{k+1}$. A contradiction now ensues by considering (hypothetically) the first such $F_{k+1}$.

\begin{ex} [5.4.3]
  Show that if $a,b,c\in F_k$ but $\sqrt{c}\notin F_k$, then $a+b\sqrt{c}=0\Leftrightarrow a=b=0$. (For $k=0$ this is in the \emph{Elements}, Book X, Prop. 79.)
\end{ex}

\begin{proof}
  Evidently, if $a=b=0$, then $a+b\sqrt{c}=0$.

  Suppose, for the sake of contradiction, that we had $b\neq0$. Then, we could conclude $\sqrt{c}=-\frac{a}{b}$. But, $-\frac{a}{b}$ is in $F_k$, a contradiction. Therefore, $b=0$ and the equation $a+b\sqrt{c}=0$ reduces to $a=0$.
\end{proof}

\begin{ex} [5.4.4]
  Suppose $\sqrt[3]{2}=a+b\sqrt{c}$, where $a,b,c\in F_k$, but that $\sqrt[3]{2}\notin F_k$. (We know that $\sqrt[3]{2}\notin F_0$ from Exercise~5.4.1.) Cube both sides and deduce from Exercise~5.4.3 that
  \[
    2=a^3+3ab^2c\quad\text{and}\quad0=3a^2b+b^3c.
  \]
\end{ex}

\begin{proof}
  \begin{align*}
    \sqrt[3]{2} &= a+b\sqrt{c} \\
    2           &= (a+b\sqrt{c})^3 \\
                &= a^3+3a^2b\sqrt{c}+3ab^2c+b^3c\sqrt{c} \\
                &= (a^3+3ab^2c)+(3a^2b+b^3c)\sqrt{c} \\
  \end{align*}
  That is, we can write
  \[
    (a^3+3ab^2c-2)+(3a^2b+b^3c)\sqrt{c}=0.
  \]
  By the previous exercise, $a^3+3ab^2c-2=0$ and $3a^2b+b^3c=0$.
\end{proof}

\begin{ex} [5.4.5]
  Deduce from Exercise~5.4.4 that $\sqrt[3]{2}=a-b\sqrt{c}$ also, and explain why this is a contradiction.
\end{ex}

\begin{proof}
  Notice that
  \[
    2=a^3+3a(-b)^2c\quad\text{and}\quad0=3a^2(-b)+(-b)^3c.
  \]
  So, $(a-b\sqrt{c})^3=2$. Since $\sqrt[3]{2}=a+b\sqrt{c}=a-b\sqrt{c}$,
  \[
    0=a+b\sqrt{c}-(a-b\sqrt{c})=2b\sqrt{c}.
  \]
  So, $b\sqrt{c}=0$ and $\sqrt[3]{2}=a\in F_k$, contradiction.
\end{proof}

\end{document}

