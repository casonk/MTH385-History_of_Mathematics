\documentclass[12pt]{article}
\usepackage{amscd,amssymb,amsthm,amsxtra,exscale,latexsym,verbatim,paralist}
\usepackage{mathrsfs}
\usepackage[T1]{fontenc}
\usepackage{newtxmath,newtxtext}
\usepackage[left = 2cm, top = 2cm, bottom = 2cm, right = 2cm]{geometry}

\usepackage{hyperref}
\usepackage{tikz}
\usetikzlibrary{patterns}

\newcommand{\XB}{\color{black}}
\newcommand{\XBB}{\color{blue}}
\newcommand{\XV}{\color{violet}}
\newcommand{\XR}{\color{red}}
\newcommand{\ds}{\displaystyle}

\setlength{\parindent}{0pt} 
\setlength{\parskip}{\baselineskip}

\theoremstyle{plain}
\newtheorem{ex}{Exercise}

\renewcommand{\proofname}{Solution}

\begin{document}

\title{\textbf{MTH385}: History of Mathematics - Homework \#7}
\date{\today}
\author{\XV\textit{\large{\href{https://github.com/casonk}{Cason Konzer}}}\XB}

\maketitle

\hrulefill

\newpage

%%%%%%%%%%%%%%%%%%%%%%%%%%%%%%%%%%%%%%%%%%%%%%%%%%%%%%%%%%%%%%%%%%%%%%%%%%%%%%%%%%%%%%%%%%%%%%%%%%%%%%%%%%%%%
%%%%%%%%%%%%%%%%%%     #1     %%%%%%%%%%%%%%%%%%%%%%%%%%%%%%%%%%%%%%%%%%%%%%%%%%%%%%%%%%%%%%%%%%%%%%%%%%%%%%%
%%%%%%%%%%%%%%%%%%%%%%%%%%%%%%%%%%%%%%%%%%%%%%%%%%%%%%%%%%%%%%%%%%%%%%%%%%%%%%%%%%%%%%%%%%%%%%%%%%%%%%%%%%%%%

Consider the intersection of two circles. Fortunately, it is easy to reduce these two quadratic equations to the case handled in Exercise~5.3.4.

\XBB\hrulefill\XB \\
\begin{ex} [5.3.5]
  The equations of any two circles can be written in the form
  \begin{align*}
    (x - a)^{2} + (y - b)^{2} &= r^{2} \\
    (x - c)^{2} + (y - d)^{2} &= s^{2}
  \end{align*}
  Explain why. Now subtract one of these equations from the other, and hence show that their common solutions can be found by rational operations and square roots.
\end{ex}
\XBB\hrulefill\XB \\

\begin{proof}
  \ \\

  The form given describes circles with center $ (a, b), (c, d) $ and radius $ r, s $, as all circles have a center and radius, they can be written in this form.

  We will first expand the squares and then subtract our two equations \dots

  \begin{itemize}
    \item $ \ds x^{2} - 2ax + a^{2} + y^{2} - 2by + b^{2} = r^{2} $.
    \item $ \ds x^{2} - 2cx + c^{2} + y^{2} - 2dy + d^{2} = s^{2} $.
    \item $ \ds 2x(c - a) + 2y(c - b) + (c^{2} + d^{2} - a^{2} - b^{2}) = r^{2} - s^{2} $.
  \end{itemize}

  Invoking the quadratic equation \dots

  \begin{itemize}
    \item $ \ds 2x(c - a) + 2y(c - b) + (c^{2} + d^{2} + s^{2} - a^{2} - b^{2} - r^{2}) = 0 $.
  \end{itemize}

  As subtracting these equations is equivalent to setting them equal, we have found a solution to their intersection which consists of only rational operations and square roots, 
  as we only perform rational operations and the square roots of the coefficents must exist to have pur centers and radai. 

\end{proof}

\newpage
%%%%%%%%%%%%%%%%%%%%%%%%%%%%%%%%%%%%%%%%%%%%%%%%%%%%%%%%%%%%%%%%%%%%%%%%%%%%%%%%%%%%%%%%%%%%%%%%%%%%%%%%%%%%%
%%%%%%%%%%%%%%%%%%     #2     %%%%%%%%%%%%%%%%%%%%%%%%%%%%%%%%%%%%%%%%%%%%%%%%%%%%%%%%%%%%%%%%%%%%%%%%%%%%%%%
%%%%%%%%%%%%%%%%%%%%%%%%%%%%%%%%%%%%%%%%%%%%%%%%%%%%%%%%%%%%%%%%%%%%%%%%%%%%%%%%%%%%%%%%%%%%%%%%%%%%%%%%%%%%%

When a sequence of quadratic equations is solved, the solution may involve \emph{nested} square roots, such as $ \sqrt{(5 + \sqrt{5}) / 2} $. This very number, in fact, occurs in the icosahedron, as one sees from Pacioli's construction in Section~2.2.

\XBB\hrulefill\XB \\
\begin{ex} [5.3.6]
  Show that the diagonal of a golden rectangle (which is also the diameter of an icosahedron of edge length $ 1 $) is $ \sqrt{(5 + \sqrt{5}) / 2} $.
\end{ex}
\XBB\hrulefill\XB \\

\begin{proof}
  \ \\

  Consider the \textit{golden rectangle}. \\
  \begin{center}
    \begin{tikzpicture}
      \draw (-3.23,-2) -- (-3.23,2);
      \draw (3.23,-2) -- (3.23,2);
      \draw (-3.23,-2) -- (3.23,-2);
      \draw (-3.23,2) -- (3.23,2);
      \node [above] at (0,2) {$ \ds \frac{1 + \sqrt{5}}{2} $};
      \node [right] at (3.23,0) {$1$};
    \end{tikzpicture}
  \end{center}

  The diagonal is the hypotenus of a right triangle thus we will invoke the Pythagorean theorem.

  \begin{itemize}
    \item $ \ds D^{2} = 1^{2} + \Bigl( \frac{1 + \sqrt{5}}{2} \Bigr)^{2} =  1 + \frac{1 + 2\sqrt{5} + 5}{4} = \frac{5 + \sqrt{5}}{2} $.
    \item $ \ds D = \sqrt{(5 + \sqrt{5})/2} $.
  \end{itemize}

  As the \textit{golden rectangle} is used to construct the edges of the icosahedron of edge length $ 1 $ the diagonals are equivalent.

\end{proof}

\newpage

%%%%%%%%%%%%%%%%%%%%%%%%%%%%%%%%%%%%%%%%%%%%%%%%%%%%%%%%%%%%%%%%%%%%%%%%%%%%%%%%%%%%%%%%%%%%%%%%%%%%%%%%%%%%%
%%%%%%%%%%%%%%%%%%     #3     %%%%%%%%%%%%%%%%%%%%%%%%%%%%%%%%%%%%%%%%%%%%%%%%%%%%%%%%%%%%%%%%%%%%%%%%%%%%%%%
%%%%%%%%%%%%%%%%%%%%%%%%%%%%%%%%%%%%%%%%%%%%%%%%%%%%%%%%%%%%%%%%%%%%%%%%%%%%%%%%%%%%%%%%%%%%%%%%%%%%%%%%%%%%%

We know from Exercise~5.4.1 that $ \sqrt[3]{2} $ is not in $ F_{0} $, but if it is constructible it will occur in some $ F_{k + 1} $. A contradiction now ensues by considering (hypothetically) the first such $ F_{k + 1} $.

\XBB\hrulefill\XB \\
\begin{ex} [5.4.3]
  Show that if $ a, b, c \in F_{k} $ but $ \sqrt{c} \notin F_{k} $, then $ a + b\sqrt{c} = 0 \Leftrightarrow a = b = 0 $. (For $ k = 0 $ this is in the \emph{Elements}, Book X, Prop. 79.)
\end{ex}
\XBB\hrulefill\XB \\

\begin{proof}
  \ \\

  For arbitrary $ k $ \dots

  \begin{itemize}
    \item $ \ds a + b\sqrt{c} = 0 \Rightarrow a = -b\sqrt{c} \Rightarrow a \notin F_{k}, \text{ as } \sqrt{c} \notin F_{k} \Rightarrow a = b = 0 $.
    \item $ \ds a = b = 0 \Rightarrow a + b\sqrt{c} = 0 + 0\sqrt{c} = 0 $.
  \end{itemize}

  The first direction leverages that $ \sqrt{c} $ would need exist in the same field as $ a $ for $ a = -b\sqrt{c} $ to hold. \\
  The second direction follows elementary algebra.

\end{proof}

\newpage

%%%%%%%%%%%%%%%%%%%%%%%%%%%%%%%%%%%%%%%%%%%%%%%%%%%%%%%%%%%%%%%%%%%%%%%%%%%%%%%%%%%%%%%%%%%%%%%%%%%%%%%%%%%%%
%%%%%%%%%%%%%%%%%%     #4     %%%%%%%%%%%%%%%%%%%%%%%%%%%%%%%%%%%%%%%%%%%%%%%%%%%%%%%%%%%%%%%%%%%%%%%%%%%%%%%
%%%%%%%%%%%%%%%%%%%%%%%%%%%%%%%%%%%%%%%%%%%%%%%%%%%%%%%%%%%%%%%%%%%%%%%%%%%%%%%%%%%%%%%%%%%%%%%%%%%%%%%%%%%%%

\XBB\hrulefill\XB \\
\begin{ex} [5.4.4]
  Suppose $ \sqrt[3]{2} = a + b\sqrt{c} $, where $ a, b, c \in F_{k} $, but that $ \sqrt[3]{2} \notin F_{k} $. (We know that $ \sqrt[3]{2} \notin F_{0} $ from Exercise~5.4.1.) Cube both sides and deduce from Exercise~5.4.3 that
  \[
    2 = a^{3} + 3ab^{2}c \quad \text{and} \quad 0 = 3a^{2}b + b^{3}c.
  \]
\end{ex}
\XBB\hrulefill\XB \\

\begin{proof}
  \ \\

  We will first expand and then group like terms.

  \begin{itemize}
    \item $ \ds (a + b\sqrt{c})^{3} = (a^{2} + 2ab\sqrt{c} + b^{2}c)(a+b\sqrt{c}) = a^{3} + 2a^{2}b\sqrt{c} + ab^{2}c + a^{2}b\sqrt{c} + 2ab^{2}c + b^{3}c\sqrt{c} $.
    \item $ \ds 2 = (\sqrt[3]{2})^{3} = (a^{3} + 3ab^{2}c) + \sqrt{c}(3a^{2}b + b^{3}c) $.
    \item $ \ds 2 \in F_{k} \Rightarrow (a^{3} + 3ab^{2}c) + \sqrt{c}(3a^{2}b + b^{3}c) \in F_{k} $.
    \item $ \ds \sqrt{c} \notin F_{k} \Rightarrow (3a^{2}b + b^{3}c) = 0 $.
  \end{itemize}

  As $ 2 $ is integer, we know $ 2 $ is in all $ F_{k} $. 
  Thus our right hand side must be in the taken $ F_{k} $. 
  But if the coefficents on $ \sqrt{c} $ is non zero, then it is implied that $ \sqrt{c} \in F_{k} $, which is a contradiction.
  Thus the coefficent on $ \sqrt{c} $ must be zero, and $ 2 = a^{3} + 3ab^{2}c $ alone.

\end{proof}

\newpage
%%%%%%%%%%%%%%%%%%%%%%%%%%%%%%%%%%%%%%%%%%%%%%%%%%%%%%%%%%%%%%%%%%%%%%%%%%%%%%%%%%%%%%%%%%%%%%%%%%%%%%%%%%%%%
%%%%%%%%%%%%%%%%%%     #5     %%%%%%%%%%%%%%%%%%%%%%%%%%%%%%%%%%%%%%%%%%%%%%%%%%%%%%%%%%%%%%%%%%%%%%%%%%%%%%%
%%%%%%%%%%%%%%%%%%%%%%%%%%%%%%%%%%%%%%%%%%%%%%%%%%%%%%%%%%%%%%%%%%%%%%%%%%%%%%%%%%%%%%%%%%%%%%%%%%%%%%%%%%%%%

\XBB\hrulefill\XB \\
\begin{ex} [5.4.5]
  Deduce from Exercise~5.4.4 that $ \sqrt[3]{2} = a - b\sqrt{c} $ also, and explain why this is a contradiction.
\end{ex}
\XBB\hrulefill\XB \\

\begin{proof}
  \ \\

  We will first expand and then group like terms.

  \begin{itemize}
    \item $ \ds (a - b\sqrt{c})^{3} = (a^{2} - 2ab\sqrt{c} + b^{2}c)(a - b\sqrt{c}) = a^{3} - 2a^{2}b\sqrt{c} + ab^{2}c - a^{2}b\sqrt{c} + 2ab^{2}c - b^{3}c\sqrt{c} $.
    \item $ \ds 2 = (\sqrt[3]{2})^{3} = (a^{3} + 3ab^{2}c) - \sqrt{c}(3a^{2}b + b^{3}c) $.
    \item $ \ds 2 \in F_{k} \Rightarrow (a^{3} + 3ab^{2}c) - \sqrt{c}(3a^{2}b + b^{3}c) \in F_{k} $.
    \item $ \ds \sqrt{c} \notin F_{k} \Rightarrow (3a^{2}b + b^{3}c) = 0 $.
  \end{itemize}

  Adding the two results we will arrive at our contraditcion \dots

  \begin{itemize}
    \item $ \ds \sqrt[3]{2} + \sqrt[3]{2} = a - b\sqrt{c} + a + b\sqrt{c} = 2a \Rightarrow \sqrt[3]{2} = a \Rightarrow \sqrt[3]{2} \in F_{k} $.
  \end{itemize}

  As $ a $ is taken in $ F_{k} $, and $ \sqrt[3]{2} $ is given not in $ F_{k} $, our equations implying that $ \sqrt[3]{2} $ is in $ F_{k} $ is the contraditcion.

\end{proof}


\end{document}

