\documentclass[12pt]{article}
\usepackage{amscd,amssymb,amsthm,amsxtra,exscale,latexsym,verbatim,paralist}
\usepackage{mathrsfs}
\usepackage[T1]{fontenc}
\usepackage{newtxmath,newtxtext}
\usepackage[margin=1in]{geometry}

%\usepackage{mathtools}
%\usepackage{multicol}
\usepackage{tikz}

\pagestyle{empty} 
\setlength{\parindent}{0pt} 
\setlength{\parskip}{\baselineskip}

\theoremstyle{plain}
\newtheorem{ex}{Exercise}

\renewcommand{\proofname}{Solution}

%\makeatletter
%\renewcommand*\env@matrix[1][*\c@MaxMatrixCols c]{%
%  \hskip -\arraycolsep
%  \let\@ifnextchar\new@ifnextchar
%  \array{#1}}
%\makeatother

\begin{document}

MTH 385 \qquad Homework due 2022-03-14

Consider the intersection of two circles. Fortunately, it is easy to reduce these two quadratic equations to the case handled in Exercise~5.3.4.

\begin{ex} [5.3.5]
  The equations of any two circles can be written in the form
  \begin{align*}
    (x-a)^2+(y-b)^2 &= r^2 \\
    (x-c)^2+(y-d)^2 &= s^2
  \end{align*}
  Explain why. Now subtract one of these equations from the other, and hence show that their common solutions can be found by rational operations and square roots.
\end{ex}

When a sequence of quadratic equations is solved, the solution may involve \emph{nested} square roots, such as $\sqrt{(5+\sqrt{5})/2}$. This very number, in fact, occurs in the icosahedron, as one sees from Pacioli's construction in Section~2.2.

\begin{ex} [5.3.6]
  Show that the diagonal of a golden rectangle (which is also the diameter of an icosahedron of edge length $1$) is $\sqrt{(5+\sqrt{5})/2}$.
\end{ex}

We know from Exercise~5.4.1 that $\sqrt[3]{2}$ is not in $F_0$, but if it is constructible it will occur in some $F_{k+1}$. A contradiction now ensues by considering (hypothetically) the first such $F_{k+1}$.

\begin{ex} [5.4.3]
  Show that if $a,b,c\in F_k$ but $\sqrt{c}\notin F_k$, then $a+b\sqrt{c}=0\Leftrightarrow a=b=0$. (For $k=0$ this is in the \emph{Elements}, Book X, Prop. 79.)
\end{ex}

\begin{ex} [5.4.4]
  Suppose $\sqrt[3]{2}=a+b\sqrt{c}$, where $a,b,c\in F_k$, but that $\sqrt[3]{2}\notin F_k$. (We know that $\sqrt[3]{2}\notin F_0$ from Exercise~5.4.1.) Cube both sides and deduce from Exercise~5.4.3 that
  \[
    2=a^3+3ab^2c\quad\text{and}\quad0=3a^2b+b^3c.
  \]
\end{ex}

\begin{ex} [5.4.5]
  Deduce from Exercise~5.4.4 that $\sqrt[3]{2}=a-b\sqrt{c}$ also, and explain why this is a contradiction.
\end{ex}

\end{document}

