\documentclass[12pt]{article}
\usepackage{amscd,amssymb,amsthm,amsxtra,exscale,latexsym,verbatim,paralist}
\usepackage{mathrsfs}
\usepackage[T1]{fontenc}
\usepackage{newtxmath,newtxtext}
\usepackage[margin=1in]{geometry}

%\usepackage{mathtools}
%\usepackage{multicol}
\usepackage{tikz}

\pagestyle{empty} 
\setlength{\parindent}{0pt} 
\setlength{\parskip}{\baselineskip}

\theoremstyle{plain}
\newtheorem{ex}{Exercise}

\renewcommand{\proofname}{Solution}

%\makeatletter
%\renewcommand*\env@matrix[1][*\c@MaxMatrixCols c]{%
%  \hskip -\arraycolsep
%  \let\@ifnextchar\new@ifnextchar
%  \array{#1}}
%\makeatother

\begin{document}

MTH 385 \qquad Homework due 2022-03-21 Solutions

\begin{ex} [5.5.2]
  Use Cardano's formula to solve $y^3=2$. Do you get the obvious solution?
\end{ex}

\begin{proof}
  Evidently, $p=0$ and $q=2$, so
  \begin{align*}
    y &= \sqrt[3]{\frac{2}{2}+\sqrt{\left(\frac{2}{2}\right)^2-\left(\frac{0}{3}\right)^3}}+\sqrt[3]{\frac{2}{2}-\sqrt{\left(\frac{2}{2}\right)^2-\left(\frac{0}{3}\right)^3}} \\
      &= \sqrt[3]{2}.
  \end{align*}
  I suspect this is the so called obvious solution.
\end{proof}

\begin{ex} [5.5.3]
  Use Cardano's formula to solve $y^3=6y+6$, and check your answer by substitution.
\end{ex}

\begin{proof}
  Evidently, $p=0$ and $q=2$.
  \begin{align*}
    y &= \sqrt[3]{\frac{6}{2}+\sqrt{\left(\frac{6}{2}\right)^2-\left(\frac{6}{3}\right)^3}}+\sqrt[3]{\frac{6}{2}-\sqrt{\left(\frac{6}{2}\right)^2-\left(\frac{6}{3}\right)^3}} \\
      &= \sqrt[3]{4}+\sqrt[3]{2} \\
  \\
    (\sqrt[3]{4}+\sqrt[3]{2})^3 &= (\sqrt[3]{4})^3+3(\sqrt[3]{4})^2(\sqrt[3]{2})+3(\sqrt[3]{4})(\sqrt[3]{2})^2+(\sqrt[3]{2})^3 \\
                                &= 4+3\sqrt[3]{32}+3\sqrt[3]{16}+2 \\
                                &= 3(2\sqrt[3]{4})+3(2\sqrt[3]{2})+6 \\
                                &= 6\sqrt[3]{4}+6\sqrt[3]{2}+6 \\
                                &= 6(\sqrt[3]{4}+\sqrt[3]{2})+6
  \end{align*}
\end{proof}

\begin{ex} [5.6.2]
  Use (3) and $\sin\alpha=\cos(\pi/2-\alpha)$, $\cos\alpha=\sin(\pi/2-\alpha)$ to show that
  \[
    (\sin\theta+i\cos\theta)^n=\left\{\begin{array}{rl}
       \sin n\theta+i\cos n\theta & \mbox{when $n=4m+1$} \\
      -\sin n\theta-i\cos n\theta & \mbox{when $n=4m+3$.}
    \end{array} \right.
  \]
\end{ex}

\begin{proof}
  \begin{align*}
    (\sin\theta+i\cos\theta)^n  &= \left[\cos\left(\frac{\pi}{2}-\theta\right)+i\sin\left(\frac{\pi}{2}-\theta\right)\right]^n \\
                                &= \cos n\left(\frac{\pi}{2}-\theta\right)+i\sin n\left(\frac{\pi}{2}-\theta\right) \\
                                &= \cos\left(\frac{n\pi}{2}-n\theta\right)+i\sin\left(\frac{n\pi}{2}-n\theta\right) \\
                                &= \left(\cos\frac{n\pi}{2}\cos n\theta+\sin\frac{n\pi}{2}\sin n\theta\right)+i\left(\sin\frac{n\pi}{2}\cos n\theta-\cos\frac{n\pi}{2}\sin n\theta\right) \\
    \intertext{When $n$ is odd, $\cos\frac{n\pi}{2}=0$.}
                                &= \sin\frac{n\pi}{2}\sin n\theta+i\sin\frac{n\pi}{2}\cos n\theta
  \end{align*}
  Finally, when $n=4m+1$, $\sin\frac{n\pi}{2}=1$. And, when $n=4m+3$, $\sin\frac{n\pi}{2}=-1$.
\end{proof}

\textbf{For the rest of the solution set, we will pretend $\sin\theta+i\cos\theta$ is the principal $n$th root of $\sin\theta+i\cos\theta)^n$. That is, we choose $\sqrt[n]{(\sin\theta+i\cos\theta)^n}=\sin\theta+i\cos\theta$ rather than some other $n$th root. We will also follow the textbook's assumption that $\sqrt{\sin^2\theta-1}=\cos\theta$ and $\sqrt{\cos^2\theta-1}=\sin\theta$, freely using the reduction from (1) to (2).}

\begin{ex} [5.6.3]
  Deduce from Exercise~5.6.2 that (2) is correct for $n=4m+1$ and false for $n=4m+3$, and hence that (1) is a correct relation between $y=\sin n\theta$ and $x=\sin\theta$ only when $n=4m+1$.
\end{ex}

\begin{proof}
  When $n=4m+1$,
  \[
    \sqrt[n]{\sin n\theta+i\cos n\theta}=\sqrt[n]{(\sin\theta+i\cos\theta)^n}=\sin\theta+i\cos\theta.
  \]
  Since $n$ is odd, $\sin n\pi=\sin\pi=0$ and
  \begin{align*}
    \frac{1}{2}\sqrt[n]{\sin n\theta+i\cos n\theta}+\frac{1}{2}\sqrt[n]{\sin n\theta-i\cos n\theta} &= \frac{1}{2}\sqrt[n]{\sin n\theta+i\cos n\theta}+\frac{1}{2}\sqrt[n]{\sin n(\pi-\theta)+i\cos n(\pi-\theta)} \\
      &= \frac{1}{2}(\sin\theta+i\cos\theta)+\frac{1}{2}(\sin(\pi-\theta)+i\cos(\pi-\theta)) \\
      &= \frac{1}{2}(\sin\theta+i\cos\theta)+\frac{1}{2}(\sin\theta-i\cos\theta) \\
      &= \sin\theta.
  \end{align*}
  When $n=4m+3$, (since $n$ is odd, $\sqrt[n]{-1}=-1$)
  \[
    \sqrt[n]{\sin n\theta+i\cos n\theta}=\sqrt[n]{-(\sin\theta+i\cos\theta)^n}=-(\sin\theta+i\cos\theta).
  \]
  Thus,
  \[
    \frac{1}{2}\sqrt[n]{\sin n\theta+i\cos n\theta}+\frac{1}{2}\sqrt[n]{\sin n\theta-i\cos n\theta}=-\sin\theta
  \]
  when $n=4m+3$. Hence, (1) is off by a factor of $-1$ when $n=4m+3$.
\end{proof}

\begin{ex} [5.6.4]
  Show that (1) is a correct relation between $y=\cos n\theta$ and $x=\cos\theta$ for all $n$ (de~Moivre (1730)).
\end{ex}

\begin{proof}
  Evidently, this reduces to
  \[
    \cos\theta=\frac{1}{2}\sqrt[n]{\cos n\theta+i\sin n\theta}+\frac{1}{2}\sqrt[n]{\cos n\theta-i\sin n\theta}
  \]
  Apply de~Moivre's formula.
  \[
    \sqrt[n]{\cos n\theta+i\sin n\theta}=\sqrt[n]{(\cos\theta+i\sin\theta)^n}=\cos\theta+i\sin\theta
  \]
  And,
  \begin{align*}
    \frac{1}{2}\sqrt[n]{\cos n\theta+i\sin n\theta}+\frac{1}{2}\sqrt[n]{\cos n\theta-i\sin n\theta} &= \frac{1}{2}\sqrt[n]{\cos n\theta+i\sin n\theta}+\frac{1}{2}\sqrt[n]{\cos n(-\theta)+i\sin n(-\theta)} \\
      &= \frac{1}{2}(\cos\theta+i\sin\theta)+\frac{1}{2}(\cos(-\theta)+i\sin(-\theta)) \\
      &= \frac{1}{2}(\cos\theta+i\sin\theta)+\frac{1}{2}(\cos\theta-i\sin\theta) \\
      &= \cos\theta.
  \end{align*}
\end{proof}

\end{document}

