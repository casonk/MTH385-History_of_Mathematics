\documentclass[12pt]{article}
\usepackage{amscd,amssymb,amsthm,amsxtra,exscale,latexsym,verbatim,paralist}
\usepackage{mathrsfs}
\usepackage[T1]{fontenc}
\usepackage{newtxmath,newtxtext}
\usepackage[margin=1in]{geometry}

%\usepackage{mathtools}
%\usepackage{multicol}
\usepackage{tikz}

\pagestyle{empty} 
\setlength{\parindent}{0pt} 
\setlength{\parskip}{\baselineskip}

\theoremstyle{plain}
\newtheorem{ex}{Exercise}

\renewcommand{\proofname}{Solution}

%\makeatletter
%\renewcommand*\env@matrix[1][*\c@MaxMatrixCols c]{%
%  \hskip -\arraycolsep
%  \let\@ifnextchar\new@ifnextchar
%  \array{#1}}
%\makeatother

\begin{document}

MTH 385 \qquad Homework due 2022-03-21

\begin{ex} [5.5.2]
  Use Cardano's formula to solve $y^3=2$. Do you get the obvious solution?
\end{ex}

\begin{ex} [5.5.3]
  Use Cardano's formula to solve $y^3=6y+6$, and check your answer by substitution.
\end{ex}

\begin{ex} [5.6.2]
  Use (3) and $\sin\alpha=\cos(\pi/2-\alpha)$, $\cos\alpha=\sin(\pi/2-\alpha)$ to show that
  \[
    (\sin\theta+i\cos\theta)^n=\left\{\begin{array}{rl}
       \sin n\theta+i\cos n\theta & \mbox{when $n=4m+1$} \\
      -\sin n\theta-i\cos n\theta & \mbox{when $n=4m+3$.}
    \end{array} \right.
  \]
\end{ex}

\begin{ex} [5.6.3]
  Deduce from Exercise~5.6.2 that (2) is correct for $n=4m+1$ and false for $n=4m+3$, and hence that (1) is a correct relation between $y=\sin n\theta$ and $x=\sin\theta$ only when $n=4m+1$.
\end{ex}

\begin{ex} [5.6.4]
  Show that (1) is a correct relation between $y=\cos n\theta$ and $x=\cos\theta$ for all $n$ (de~Moivre (1730)).
\end{ex}

\end{document}

