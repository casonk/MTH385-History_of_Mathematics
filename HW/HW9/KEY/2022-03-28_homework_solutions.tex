\documentclass[12pt]{article}
\usepackage{amscd,amssymb,amsthm,amsxtra,exscale,latexsym,verbatim,paralist}
\usepackage{mathrsfs}
\usepackage[T1]{fontenc}
\usepackage{newtxmath,newtxtext}
\usepackage[margin=1in]{geometry}

%\usepackage{mathtools}
%\usepackage{multicol}
\usepackage{tikz}

\pagestyle{empty} 
\setlength{\parindent}{0pt} 
\setlength{\parskip}{\baselineskip}

\theoremstyle{plain}
\newtheorem{ex}{Exercise}

\renewcommand{\proofname}{Solution}

%\makeatletter
%\renewcommand*\env@matrix[1][*\c@MaxMatrixCols c]{%
%  \hskip -\arraycolsep
%  \let\@ifnextchar\new@ifnextchar
%  \array{#1}}
%\makeatother

\begin{document}

MTH 385 \qquad Homework due 2022-03-28 Solutions

\begin{ex} [5.7.2]
  If $p(x)=a_kx^k+a_{k-1}x^{k-1}+\cdots+a_1x+a_0$, use Exercise~5.7.1 to show that $p(x)-p(a)$ has a factor $x-a$.
\end{ex}

\begin{proof}
  In class, we did Exercise~5.7.1. So, we know that, for all positive integers $m$,
  \begin{align*}
    (x-a) & (x^{m-1}+ax^{m-2}+\cdots+a^{m-2}x+a^{m-1}) \\
      &= (x^m-ax^{m-1})+(ax^{m-1}-a^2x^{m-2})+\cdots+(a^{m-2}x^2-a^{m-1}x)+(a^{m-1}x-a^m) \\
      &= x^m-a^m.
  \end{align*}
  Evidently,
  \begin{align*}
    p(x)-p(a) &= (a_kx^k+a_{k-1}x^{k-1}+\cdots+a_1x+a_0)-(a_ka^k+a_{k-1}a^{k-1}+\cdots+a_1a+a_0) \\
              &= a_k(x^k-a^k)+a_{k-1}(x^{k-1}-a^{k-1})+\cdots+a_1(x-a)
  \end{align*}
  is divisible by $x-a$.
\end{proof}

\begin{ex} [5.7.3]
  Deduce Descartes's theorem from Exercise~5.7.2.
\end{ex}

\begin{proof}
  Look at Exercise~5.7.2. If we suppose $p(a)=0$, then $x-a$ is a factor of
  \[
    p(x)-p(a)=p(x).
  \]
\end{proof}

Recall
\[
  \binom{n}{k}=\binom{n-1}{k-1}+\binom{n-1}{k}.
\]

This property gives an easy way to calculate Pascal's triangle to any depth, and hence compute a fair division of stakes in a game that has to be called off with $n$ plays remaining. We suppose that players I and II have an equal chance of winning each play, and that I needs to win $k$ of the remaining $n$ plays to carry off the stakes.

\begin{ex} [5.8.2]
  Show that the ratio of I's winning the stakes to that of II's winning is
  \[
    \binom{n}{n}+\binom{n}{n-1}+\cdots+\binom{n}{k}:\binom{n}{k-1}+\binom{n}{k-2}+\cdots+\binom{n}{0}.
  \]
\end{ex}

\begin{proof}
  Encode the possible sequences of the remaining plays as strings of the length $n$ consisting of $1$s and $2$s, where the $k$th entry in a given string is $1$ if player I won the $k$th play and $2$ if player II won the $k$th play. Since each player has an equal chance of winning each play, each of the sequences of $n$ plays is equally likely. So,the ratio of I's winning the stakes to that of II's winning is the number of our strings that contain at least $k$ $1$s divided by the number of our strings that contain fewer than $k$ $1$s. Moreover, the number of our strings that contain exactly $m$ $1$s ones is $\binom{n}{m}$ since constructing such a string is equivalent to choosing the $m$ locations in the string to place the $1$s from the possible $n$ locations in the string.
\end{proof}

The sum property of the binomial coefficients also explains the presence of some interesting numbers in Pascal's triangle.

\begin{ex} [5.8.3]
  Explain why the third diagonal from the left in the triangle, namely \\
  $1,3,6,10,15,21,\ldots$, consists of the triangular numbers.
\end{ex}

\begin{proof}
  For the next two exercises, let the $n$th triangular number be
  \[
    T_n=\left|\left\{\begin{bmatrix}x\\y\\z\end{bmatrix}\in\mathbb{Z}^3\,\middle|\,x,y,z\geq0,\,x+y+z=n-1\right\}\right|.
  \]
  That is,
  \[
    T_1=\left|\left\{\begin{bmatrix}0\\0\\0\end{bmatrix}\right\}\right|=1,\quad
    T_2=\left|\left\{\begin{bmatrix}1\\0\\0\end{bmatrix},\begin{bmatrix}0\\1\\0\end{bmatrix},\begin{bmatrix}0\\0\\1\end{bmatrix}\right\}\right|=3,\quad
    T_3=\left|\left\{\begin{bmatrix}2\\0\\0\end{bmatrix},\begin{bmatrix}1\\1\\0\end{bmatrix},\begin{bmatrix}0\\2\\0\end{bmatrix},\begin{bmatrix}1\\0\\1\end{bmatrix},\begin{bmatrix}0\\1\\1\end{bmatrix},\begin{bmatrix}0\\0\\2\end{bmatrix}\right\}\right|=6,\ldots
  \]
  This is the number of lattice (integer coordinate) points in the triangle
  \[
    \left\{\begin{bmatrix}x\\y\\z\end{bmatrix}\in\mathbb{R}^3\,\middle|\,x,y,z\geq0,\,x+y+z=n-1\right\}.
  \]
  When we group the points in the triangles by decreasing $z$-coordinate, we see
  \[
    T_n=\sum_{k=1}^nk.
  \]
  I claim $T_n=\binom{n+1}{2}$. We will prove this by induction. We showed above that $\displaystyle T_1=1=\binom{2}{2}$. Now, suppose $\displaystyle T_{m-1}=\binom{n}{2}$ and consider $T_m$
  \[
    T_m=m+T_{m-1}=\binom{m}{1}+\binom{m}{2}=\binom{m+1}{2}\qquad\left(\text{Recall }\binom{m}{1}=m\text{ for all }m.\right)
  \]
\end{proof}

\begin{ex} [5.8.4]
  The numbers on the next diagonal, namely $1,4,10,20,35\ldots$, can be called \emph{tetrahedral numbers}. Why is this an apt description?
\end{ex}

\begin{proof}
  Consider $\displaystyle t_n=\binom{n+2}{3}$. Notice that when $n>1$,
  \begin{align*}
    t_n &= \binom{n+2}{3} \\
      &= \binom{n+1}{2}+\binom{n+1}{3} \\
      &= T_n+t_{n-1} \\
      &= \left|\left\{\begin{bmatrix}x\\y\\z\end{bmatrix}\in\mathbb{Z}^3\,\middle|\,x,y,z\geq0,\,x+y+z\leq n-1\right\}\right|.
  \end{align*}
  This is the number of lattice points in the tetrahedron
  \[
    \left\{\begin{bmatrix}x\\y\\z\end{bmatrix}\in\mathbb{R}^3\,\middle|\,x,y,z\geq0,\,x+y+z\leq n-1\right\}.
  \]
\end{proof}

\end{document}

