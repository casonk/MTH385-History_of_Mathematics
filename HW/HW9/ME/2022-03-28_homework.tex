\documentclass[12pt]{article}
\usepackage{amscd,amssymb,amsthm,amsxtra,exscale,latexsym,verbatim,paralist}
\usepackage{mathrsfs}
\usepackage[T1]{fontenc}
\usepackage{newtxmath,newtxtext}
\usepackage[left = 2cm, top = 2cm, bottom = 2cm, right = 2cm]{geometry}

\usepackage{hyperref}
\usepackage{tikz}
\usetikzlibrary{patterns}

\usepackage{pgfplots}
\pgfplotsset{compat = newest}

\newcommand{\XB}{\color{black}}
\newcommand{\XBB}{\color{blue}}
\newcommand{\XV}{\color{violet}}
\newcommand{\XR}{\color{red}}
\newcommand{\ds}{\displaystyle}

\setlength{\parindent}{0pt} 
\setlength{\parskip}{\baselineskip}

\theoremstyle{plain}
\newtheorem{ex}{Exercise}

\renewcommand{\proofname}{Solution}

\begin{document}

\title{\textbf{MTH385}: History of Mathematics - Homework \#9}
\date{\today}
\author{\XV\textit{\large{\href{https://github.com/casonk}{Cason Konzer}}}\XB}

\maketitle

\hrulefill

\newpage

%%%%%%%%%%%%%%%%%%%%%%%%%%%%%%%%%%%%%%%%%%%%%%%%%%%%%%%%%%%%%%%%%%%%%%%%%%%%%%%%%%%%%%%%%%%%%%%%%%%%%%%%%%%%%
%%%%%%%%%%%%%%%%%%     #1     %%%%%%%%%%%%%%%%%%%%%%%%%%%%%%%%%%%%%%%%%%%%%%%%%%%%%%%%%%%%%%%%%%%%%%%%%%%%%%%
%%%%%%%%%%%%%%%%%%%%%%%%%%%%%%%%%%%%%%%%%%%%%%%%%%%%%%%%%%%%%%%%%%%%%%%%%%%%%%%%%%%%%%%%%%%%%%%%%%%%%%%%%%%%%

\XBB\hrulefill\XB \\
\begin{ex} [5.7.2]
  If $ p(x) = a_{k}x^{k} + a_{k - 1}x^{k - 1} + \cdots + a_{1}x + a_{0} $, use Exercise~5.7.1 to show that $ p(x) - p(a) $ has a factor $ x - a $.
\end{ex}
\XBB\hrulefill\XB \\

\begin{proof}
  \ \\

  \begin{itemize}
    \item $ \ds p(x) = a_{k}x^{k} + a_{k - 1}x^{k - 1} + \cdots + a_{1}x + a_{0} $.
    \item $ \ds p(a) = a_{k}a^{k} + a_{k - 1}a^{k - 1} + \cdots + a_{1}a + a_{0} $.
    \item $ \ds p(x) - p(a) = a_{k}(x^{k} - a^{k}) + a_{k - 1}(x^{k - 1} - a^{k - 1}) + \cdots + a_{2}(x^{2} - a^{2}) + a_{1}(x - a) $.
  \end{itemize}

  \begin{itemize}
    \item $ \ds (x^{k} - a^{k}) / (x - a) =  (x^{k - 1} + ax^{k - 2} + \cdots + a^{k - 2}x + a^{k - 1}) = C $.
    \item $ \ds (x^{k - 1} - a^{k - 1}) / (x - a) = (x^{k - 2} + ax^{k - 3} + \cdots + a^{k - 3}x + a^{k - 2}) = A $.
    \item $ \ds  \dots / (x - a) = \dots = S $.
    \item $ \ds (x^{2} - a^{2}) / (x - a) = (x + a) = O $.
    \item $ \ds (x - a) / (x - a) = 1 = N$.
  \end{itemize}

  \begin{itemize}
    \item $ \ds (p(x) - p(a)) / (x - a) = C + A + S + O + N $.
  \end{itemize}

\end{proof}

\newpage
%%%%%%%%%%%%%%%%%%%%%%%%%%%%%%%%%%%%%%%%%%%%%%%%%%%%%%%%%%%%%%%%%%%%%%%%%%%%%%%%%%%%%%%%%%%%%%%%%%%%%%%%%%%%%
%%%%%%%%%%%%%%%%%%     #2     %%%%%%%%%%%%%%%%%%%%%%%%%%%%%%%%%%%%%%%%%%%%%%%%%%%%%%%%%%%%%%%%%%%%%%%%%%%%%%%
%%%%%%%%%%%%%%%%%%%%%%%%%%%%%%%%%%%%%%%%%%%%%%%%%%%%%%%%%%%%%%%%%%%%%%%%%%%%%%%%%%%%%%%%%%%%%%%%%%%%%%%%%%%%%

\XBB\hrulefill\XB \\
\begin{ex} [5.7.3]
  Deduce Descartes's theorem from Exercise~5.7.2.
\end{ex}
\XBB\hrulefill\XB \\

\begin{proof}
  \ \\

  If $ p(a) = 0 $ Then \dots

  \begin{itemize}
    \item $ \ds (p(x) - p(a)) / (x - a) = p(x) / (x - a) = C + A + S + O + N = K $.
  \end{itemize}

  Thus $ p(x) $, with value $ 0 $ when $ x = a $, has a factor $ (x - a) $. 

  As the largest degree present in $ K $, found in $ C $, is $ k - 1 $, 
  we are left with a polynomial of degree $ k - 1 $ when dividing the polynomial $ (p(x) - p(a)) = p(x) $ of degree $ k $ by $ (x - a) $.
\end{proof}

\newpage

Recall
\[
  \binom{n}{k} = \binom{n - 1}{k - 1} + \binom{n - 1}{k}.
\]

This property gives an easy way to calculate Pascal's triangle to any depth, 
and hence compute a fair division of stakes in a game that has to be called off with $ n $ plays remaining. 
We suppose that players I and II have an equal chance of winning each play, 
and that I needs to win $ k $ of the remaining $ n $ plays to carry off the stakes.

%%%%%%%%%%%%%%%%%%%%%%%%%%%%%%%%%%%%%%%%%%%%%%%%%%%%%%%%%%%%%%%%%%%%%%%%%%%%%%%%%%%%%%%%%%%%%%%%%%%%%%%%%%%%%
%%%%%%%%%%%%%%%%%%     #3     %%%%%%%%%%%%%%%%%%%%%%%%%%%%%%%%%%%%%%%%%%%%%%%%%%%%%%%%%%%%%%%%%%%%%%%%%%%%%%%
%%%%%%%%%%%%%%%%%%%%%%%%%%%%%%%%%%%%%%%%%%%%%%%%%%%%%%%%%%%%%%%%%%%%%%%%%%%%%%%%%%%%%%%%%%%%%%%%%%%%%%%%%%%%%

\XBB\hrulefill\XB \\
\begin{ex} [5.8.2]
  Show that the ratio of I's winning the stakes to that of II's winning is
  \[
    \binom{n}{n} + \binom{n}{n - 1} + \cdots + \binom{n}{k} : \binom{n}{k - 1} + \binom{n}{k - 2} + \cdots + \binom{n}{0}.
  \]
\end{ex}
\XBB\hrulefill\XB \\

\begin{proof}
  \ \\

  We must assume that if player I does not win, then Player II wins \dots

  We can then think of all results in terms of Player I's outcome.

  If Player I wins $ i $ times, where $ k \le i \le n $, then Player I wins the stakes.

  If Player I wins $ j $ times, where $ 0 \le j < k $, then Player II wins the stakes.

  Thus there is $ \ds \sum_{i = k}^{n} \binom{n}{i} $ ways for Player I to win the stakes.

  Similarly there is $ \ds \sum_{j = 0}^{k-1} \binom{n}{j} $ ways for Player II to win the stakes.

  Our ratio, $ P_{I_{wins \ stakes}} : P_{II_{wins \ stakes}} $, is then $ \ds \sum_{i = k}^{n} \binom{n}{i} : \sum_{j = 0}^{k-1} \binom{n}{j} $, as asked to be shown. 
\end{proof}

\newpage

The sum property of the binomial coefficients also explains the presence of some interesting numbers in Pascal's triangle.

%%%%%%%%%%%%%%%%%%%%%%%%%%%%%%%%%%%%%%%%%%%%%%%%%%%%%%%%%%%%%%%%%%%%%%%%%%%%%%%%%%%%%%%%%%%%%%%%%%%%%%%%%%%%%
%%%%%%%%%%%%%%%%%%     #4     %%%%%%%%%%%%%%%%%%%%%%%%%%%%%%%%%%%%%%%%%%%%%%%%%%%%%%%%%%%%%%%%%%%%%%%%%%%%%%%
%%%%%%%%%%%%%%%%%%%%%%%%%%%%%%%%%%%%%%%%%%%%%%%%%%%%%%%%%%%%%%%%%%%%%%%%%%%%%%%%%%%%%%%%%%%%%%%%%%%%%%%%%%%%%

\XBB\hrulefill\XB \\
\begin{ex} [5.8.3]
  Explain why the third diagonal from the left in the triangle, namely \\
  $ 1, 3, 6, 10, 15, 21, \ldots $, consists of the triangular numbers.
\end{ex}
\XBB\hrulefill\XB \\

\begin{proof}
  \ \\

  Note first that the triangular numbers take the form, $ \ds T_{t} = \sum_{i = 1}^{t} i $.

  Now note that this diagonal referenced represents the binomial coefficients $ \ds \binom{n}{2} $ such that $ n \ge 2 $. 

  Considering an arbitrany n, we have $ \ds \binom{n}{2} = \binom{n - 1}{1} + \binom{n - 1}{2} $. 
  
  This process is then repeated until we arrive at $ \ds \binom{n}{2} = \binom{n - 1}{1} + \binom{n - 2}{1} + \dots + \binom{3}{1} + \binom{2}{1} + \binom{2}{2} $.

  These combinations take the integer values : $ n - 1, n - 2, \dots, 3, 2, 1 $.

  Thus we have that $ \ds \binom{n}{2} = \sum_{i = 1}^{n - 1} i = T_{n - 1} $, hence why the diagonal consists of the triangular numbers.
\end{proof}

\newpage

%%%%%%%%%%%%%%%%%%%%%%%%%%%%%%%%%%%%%%%%%%%%%%%%%%%%%%%%%%%%%%%%%%%%%%%%%%%%%%%%%%%%%%%%%%%%%%%%%%%%%%%%%%%%%
%%%%%%%%%%%%%%%%%%     #5     %%%%%%%%%%%%%%%%%%%%%%%%%%%%%%%%%%%%%%%%%%%%%%%%%%%%%%%%%%%%%%%%%%%%%%%%%%%%%%%
%%%%%%%%%%%%%%%%%%%%%%%%%%%%%%%%%%%%%%%%%%%%%%%%%%%%%%%%%%%%%%%%%%%%%%%%%%%%%%%%%%%%%%%%%%%%%%%%%%%%%%%%%%%%%

\XBB\hrulefill\XB \\
\begin{ex} [5.8.4]
  The numbers on the next diagonal, namely $ 1, 4, 10, 20, 35 \ldots $, can be called \emph{tetrahedral numbers}. Why is this an apt description?
\end{ex}
\XBB\hrulefill\XB \\

\begin{proof}
  \ \\

  This description is apt as these numbers take the from $ \ds TT_{t} = \sum_{i = 1}^{t} T_{i} $.

  e.g. $ 1 = 1; \ 4 = 1 + 3; \ 10 = 1 + 3 + 6; \ 20 = 1 + 3 + 6 + 10; \ 35 = 1 + 3 + 6 + 10 + 15 $ \dots

  The numbers can be visualized in a similar manner to that of the triangular numbers, 
  such that $ TT_{t} $ is a triangular pyramid with base $ T_{t} $ and each incremental level above the base, 
  or the level below it, is the triangluar number which is index as one below, until finally reaching $ T_{1} $, 
  the single tip of the triangular pyramid.

\end{proof}

\newpage

\end{document}

