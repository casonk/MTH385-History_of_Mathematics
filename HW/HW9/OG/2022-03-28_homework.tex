\documentclass[12pt]{article}
\usepackage{amscd,amssymb,amsthm,amsxtra,exscale,latexsym,verbatim,paralist}
\usepackage{mathrsfs}
\usepackage[T1]{fontenc}
\usepackage{newtxmath,newtxtext}
\usepackage[margin=1in]{geometry}

%\usepackage{mathtools}
%\usepackage{multicol}
\usepackage{tikz}

\pagestyle{empty} 
\setlength{\parindent}{0pt} 
\setlength{\parskip}{\baselineskip}

\theoremstyle{plain}
\newtheorem{ex}{Exercise}

\renewcommand{\proofname}{Solution}

%\makeatletter
%\renewcommand*\env@matrix[1][*\c@MaxMatrixCols c]{%
%  \hskip -\arraycolsep
%  \let\@ifnextchar\new@ifnextchar
%  \array{#1}}
%\makeatother

\begin{document}

MTH 385 \qquad Homework due 2022-03-28

\begin{ex} [5.7.2]
  If $p(x)=a_kx^k+a_{k-1}x^{k-1}+\cdots+a_1x+a_0$, use Exercise~5.7.1 to show that $p(x)-p(a)$ has a factor $x-a$.
\end{ex}

\begin{ex} [5.7.3]
  Deduce Descartes's theorem from Exercise~5.7.2.
\end{ex}

Recall
\[
  \binom{n}{k}=\binom{n-1}{k-1}+\binom{n-1}{k}.
\]

This property gives an easy way to calculate Pascal's triangle to any depth, and hence compute a fair division of stakes in a game that has to be called off with $n$ plays remaining. We suppose that players I and II have an equal chance of winning each play, and that I needs to win $k$ of the remaining $n$ plays to carry off the stakes.

\begin{ex} [5.8.2]
  Show that the ratio of I's winning the stakes to that of II's winning is
  \[
    \binom{n}{n}+\binom{n}{n-1}+\cdots+\binom{n}{k}:\binom{n}{k-1}+\binom{n}{k-2}+\cdots+\binom{n}{0}.
  \]
\end{ex}

The sum property of the binomial coefficients also explains the presence of some interesting numbers in Pascal's triangle.

\begin{ex} [5.8.3]
  Explain why the third diagonal from the left in the triangle, namely \\
  $1,3,6,10,15,21,\ldots$, consists of the triangular numbers.
\end{ex}

\begin{ex} [5.8.4]
  The numbers on the next diagonal, namely $1,4,10,20,35\ldots$, can be called \emph{tetrahedral numbers}. Why is this an apt description?
\end{ex}

\end{document}

