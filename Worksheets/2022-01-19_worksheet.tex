\documentclass[12pt]{article}
\usepackage{amscd,amssymb,amsthm,amsxtra,exscale,latexsym,verbatim,paralist}
\usepackage{mathrsfs}
\usepackage[T1]{fontenc}
\usepackage{newtxmath,newtxtext}
\usepackage[margin=1in]{geometry}

%\usepackage{mathtools}
%\usepackage{multicol}
\usepackage{tikz}

\pagestyle{empty} 
\setlength{\parindent}{0pt} 
\setlength{\parskip}{\baselineskip}

\theoremstyle{plain}
\newtheorem{ex}{Exercise}

\renewcommand{\proofname}{Solution}

%\makeatletter
%\renewcommand*\env@matrix[1][*\c@MaxMatrixCols c]{%
%  \hskip -\arraycolsep
%  \let\@ifnextchar\new@ifnextchar
%  \array{#1}}
%\makeatother

\begin{document}

MTH 385 \qquad 2022-01-19 Worksheet

\begin{ex}
  As you may recall, the proof that $\sqrt{2}$ is irrational is often used to teach an important proof technique. What is that proof technique? And, how does it work?
\end{ex}

\begin{ex}
  State the Well-Ordering Principle. (And, look for its appearance in the following.)
\end{ex}

\begin{ex}
  The proof recounted in our textbook uses the fact that any (positive) rational number can be written as the ratio of two realtively prime (positive) integers. Prove this fact.
\end{ex}

\begin{ex}
  The proof recounted in our textbook also uses the fact that, if the square of a positive integer is even, then that positive integer is also even. Prove this fact.
\end{ex}

\begin{ex}
  Do Exercise~1.5.1: Writing an arbitrary odd number $m$ in the form $2q+1$, for some integer $q$, show that $m^2$ also has the form $2r+1$, which shows that $m^2$ is also odd.
\end{ex}

\begin{ex}
  What happens if we replace $2$ by $3$ in Exercise~1.5.1? Write $m=3q+1$, for some integer $q$, does $m^2$ also have the form $3r+1$?
\end{ex}

\begin{ex}
  What happens if we replace $2$ by $4$ in Exercise~1.5.1? Write $m=4q+1$, for some integer $q$, does $m^2$ also have the form $4r+1$?
\end{ex}

\begin{ex}
  What happens if we replace $2$ by $5$ in Exercise~1.5.1? Write $m=5q+1$, for some integer $q$, does $m^2$ also have the form $5r+1$?
\end{ex}

\begin{ex}
  What, if anything, can be said if we replace $2$ in Exercise~1.5.1 by an arbitrary (positive) integer?
\end{ex}

\begin{ex}
  The proof that $\sqrt{2}$ is irrational is an early instance of the recurring interplay between geometry on the one hand and arithmetic/algebra on the other. Are there other examples of such interplay between streams of mathematical thought in the $100$-level and $200$-level courses at UM-Flint? If so, list some.
\end{ex}

\end{document}

