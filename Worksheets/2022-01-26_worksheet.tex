\documentclass[12pt]{article}
\usepackage{amscd,amssymb,amsthm,amsxtra,exscale,latexsym,verbatim,paralist}
\usepackage{mathrsfs}
\usepackage[T1]{fontenc}
\usepackage{newtxmath,newtxtext}
\usepackage[margin=1in]{geometry}

%\usepackage{mathtools}
%\usepackage{multicol}
\usepackage{tikz}

\pagestyle{empty} 
\setlength{\parindent}{0pt} 
\setlength{\parskip}{\baselineskip}

\theoremstyle{plain}
\newtheorem{ex}{Exercise}

\renewcommand{\proofname}{Solution}

%\makeatletter
%\renewcommand*\env@matrix[1][*\c@MaxMatrixCols c]{%
%  \hskip -\arraycolsep
%  \let\@ifnextchar\new@ifnextchar
%  \array{#1}}
%\makeatother

\begin{document}

MTH 385 \qquad 2022-01-26 Worksheet

\begin{ex}
  Define the following terms: (convex) polygon, (convex) polyhedron, vertex, edge, face.
\end{ex}

\begin{ex}
  What is a \emph{regular polyhedron}?
\end{ex}

\begin{ex}
  What kind of polygons can be faces of regular polyhedra? Why?
\end{ex}

\begin{ex}
  How many triangles can meet at a vertex of a regular polyhedron? Why?
\end{ex}

\begin{ex}
  How many quadrilaterals can meet at a vertex of a regular polyhedron? Why?
\end{ex}

\begin{ex}
  How many pentagons can meet at a vertex of a regular polyhedron? Why?
\end{ex}

\begin{ex}
  How many hexagons can meet at a vertex of a regular polyhedron? Why?
\end{ex}

\begin{ex}
  Abstractly construct a complete list of (potential) regular polyhedra based on your answers to the previous questions.
\end{ex}

\begin{ex}
  Complete the following table.
  \Large
    \begin{center}
      \begin{tabular}{|| l | c | c | c | c ||}
        \hline\hline
        Regular Polyhedron & Vertices ($V$) & Edges ($E$) & Faces ($F$) & $V-E+F$ \\ \hline
                           &                &             &             &         \\ \hline
                           &                &             &             &         \\ \hline
                           &                &             &             &         \\ \hline
                           &                &             &             &         \\ \hline
                           &                &             &             &         \\ \hline
                           &                &             &             &         \\ \hline
                           &                &             &             &         \\ \hline
                           &                &             &             &         \\ \hline
                           &                &             &             &         \\ \hline
                           &                &             &             &         \\ \hline\hline
      \end{tabular}
    \end{center}
  \normalsize 
\end{ex}

\begin{ex}
  Make observations about the table.
\end{ex}

\end{document}

