\documentclass[12pt]{article}
\usepackage{amscd,amssymb,amsthm,amsxtra,exscale,latexsym,verbatim,paralist}
\usepackage{mathrsfs}
\usepackage[T1]{fontenc}
\usepackage{newtxmath,newtxtext}
\usepackage[margin=1in]{geometry}

%\usepackage{mathtools}
%\usepackage{multicol}
\usepackage{tikz}

\pagestyle{empty} 
\setlength{\parindent}{0pt} 
\setlength{\parskip}{\baselineskip}

\theoremstyle{plain}
\newtheorem{ex}{Exercise}

\renewcommand{\proofname}{Solution}

%\makeatletter
%\renewcommand*\env@matrix[1][*\c@MaxMatrixCols c]{%
%  \hskip -\arraycolsep
%  \let\@ifnextchar\new@ifnextchar
%  \array{#1}}
%\makeatother

\begin{document}

MTH 385 \qquad 2022-01-31 Worksheet

\begin{ex}
  Describe the procedure for bisecting a given line segment using a compass and straightedge.
\end{ex}

\begin{ex}
  Describe the procedure for bisecting a given angle using a compass and straightedge.
\end{ex}

\begin{ex}
  Describe the procedure for constructing a perpendicular to a given line segment at a given point using a compass and straightedge.
\end{ex}

\begin{ex}
  Describe the procedure for drawing a circle through three (non-collinear) given points using a compass and straightedge.
\end{ex}

\begin{ex}
  Given line segments of lengths $1$, $\ell_1$, and $\ell_2$, describe the procedures for constructing line segments of the following lengths using a compass and straightedge.
  \begin{enumerate}[(a)]
    \item $\ell_1+\ell_2$
    \item $|\ell_1-\ell_2|$
    \item $\ell_1\ell_2$
    \item $\ell_1/\ell_2$
    \item $\sqrt{\ell_1}$
  \end{enumerate}
\end{ex}

\end{document}

