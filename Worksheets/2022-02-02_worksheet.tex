\documentclass[12pt]{article}
\usepackage{amscd,amssymb,amsthm,amsxtra,exscale,latexsym,verbatim,paralist}
\usepackage{mathrsfs}
\usepackage[T1]{fontenc}
\usepackage{newtxmath,newtxtext}
\usepackage[margin=1in]{geometry}

%\usepackage{mathtools}
%\usepackage{multicol}
\usepackage{tikz}

\pagestyle{empty} 
\setlength{\parindent}{0pt} 
\setlength{\parskip}{\baselineskip}

\theoremstyle{plain}
\newtheorem{ex}{Exercise}

\renewcommand{\proofname}{Solution}

%\makeatletter
%\renewcommand*\env@matrix[1][*\c@MaxMatrixCols c]{%
%  \hskip -\arraycolsep
%  \let\@ifnextchar\new@ifnextchar
%  \array{#1}}
%\makeatother

\begin{document}

MTH 385 \qquad 2022-02-02 Worksheet

\begin{ex}
  Why are the ellipse, hyperbola, and parabola called conic sections?
\end{ex}

\begin{ex}
  For an ellipse, define the following terms.
  \begin{enumerate}[(a)]
    \item center
    \item co-vertex
    \item focus
    \item major axis
    \item minor axis
    \item vertex
  \end{enumerate}
\end{ex}

\begin{ex}
  The standard equation for an ellipse is
  \[
    \frac{x^2}{a^2}+\frac{y^2}{b^2}=1.
  \]
  (Assume $a>b$.) Given an ellipse in the plane, how are the coordinate axes chosen so that an equation in this form holds? How are the constants $a$ and $b$ related to the notions in the previous exercise?
\end{ex}

\begin{ex}
  For a hyperbola, define the following terms.
  \begin{enumerate}[(a)]
    \item asymptote
    \item center
    \item co-vertex
    \item conjugate axis
    \item focus
    \item transverse axis
    \item vertex
  \end{enumerate}
\end{ex}

\begin{ex}
  A standard equation for a hyperbola is
  \[
    \frac{x^2}{a^2}-\frac{y^2}{b^2}=1.
  \]
  Given a hyperbola in the plane, how are the coordinate axes chosen so that an equation in this form holds? How are the constants $a$ and $b$ related to the notions in the previous exercise?
\end{ex}

\begin{ex}
  For an parabola, define the following terms.
  \begin{enumerate}[(a)]
    \item axis of symmetry
    \item directrix
    \item focus
    \item latus rectum
    \item vertex
  \end{enumerate}
\end{ex}

\begin{ex}
  A standard equation for a parabola is
  \[
    x^2=4py.
  \]
  Given a parabola in the plane, how are the coordinate axes chosen so that an equation in this form holds? How is the constant $p$ related to the notions in the previous exercise?
\end{ex}

\begin{ex}
  Given real numbers $A$, $B$, $C$, $D$, $E$, and $F$, can we determine whether the curve
  \[
    Ax^2+Bxy+Cy^2+Dx+Ey+F=0
  \]
  is an ellipse, a hyperbola, or a parabola? If so, how?
\end{ex}

\end{document}

