\documentclass[12pt]{article}
\usepackage{amscd,amssymb,amsthm,amsxtra,exscale,latexsym,verbatim,paralist}
\usepackage{mathrsfs}
\usepackage[T1]{fontenc}
\usepackage{newtxmath,newtxtext}
\usepackage[margin=1in]{geometry}

%\usepackage{mathtools}
%\usepackage{multicol}
\usepackage{tikz}

\pagestyle{empty} 
\setlength{\parindent}{0pt} 
\setlength{\parskip}{\baselineskip}

\theoremstyle{plain}
\newtheorem{ex}{Exercise}

\renewcommand{\proofname}{Solution}

%\makeatletter
%\renewcommand*\env@matrix[1][*\c@MaxMatrixCols c]{%
%  \hskip -\arraycolsep
%  \let\@ifnextchar\new@ifnextchar
%  \array{#1}}
%\makeatother

\begin{document}

MTH 385 \qquad 2022-02-07 Worksheet

\begin{center}
  \begin{tikzpicture}%[scale=7.5]
    \draw [->] (-7,0) -- (7,0);
    \node [right] at (7,0) {$X$};
    \draw [->] (0,-7) -- (0,7);
    \node [above] at (0,7) {$Y$};
    \draw (0,0) circle [radius=5];
    \node [above left] at (0,0) {$O$};
    \draw [-] (-3,-4) -- (-3,4);
    \node [above left] at (-3,0) {$-x$};
    \draw [-] (3,-4) -- (3,4);
    \node [above right] at (3,0) {$x$};
    \draw [-] (-3,-4) -- (5,0);
    \node [above right] at (5,0) {$R$};
    \draw [fill] (3,-1) circle [radius=0.05];
    \node [below right] at (3,-1) {$P$};
    \draw [red, thick, domain=0:5] plot (\x, {-sqrt(pow(5-\x,3)/(5+\x))});
  \end{tikzpicture}

  Figure~2.8: Construction of the cissoid
\end{center}

\begin{ex}
  I don't think the author of our textbook gives an adequate description of the cissoid. I think part of the description is only implied. How exactly are the points on this curve determined?
\end{ex}

\begin{ex} [2.5.1]
  Using $X$ and $Y$ for the horizontal and vertical coordinates, show that the straight line $RP$ in Figure~2.8 has equation
  \[
    Y=\frac{\sqrt{1-x^2}}{1+x}(X-1).
  \]
\end{ex}

\begin{ex} [2.5.2]
  Deduce the equation of the cissoid from Exercise~2.5.1.
\end{ex}

\end{document}

