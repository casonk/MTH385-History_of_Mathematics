\documentclass[12pt]{article}
\usepackage{amscd,amssymb,amsthm,amsxtra,exscale,latexsym,verbatim,paralist}
\usepackage{mathrsfs}
\usepackage[T1]{fontenc}
\usepackage{newtxmath,newtxtext}
\usepackage[margin=1in]{geometry}

%\usepackage{mathtools}
%\usepackage{multicol}
\usepackage{tikz}

\pagestyle{empty} 
\setlength{\parindent}{0pt} 
\setlength{\parskip}{\baselineskip}

\theoremstyle{plain}
\newtheorem{ex}{Exercise}

\renewcommand{\proofname}{Solution}

%\makeatletter
%\renewcommand*\env@matrix[1][*\c@MaxMatrixCols c]{%
%  \hskip -\arraycolsep
%  \let\@ifnextchar\new@ifnextchar
%  \array{#1}}
%\makeatother

\begin{document}

MTH 385 \qquad 2022-02-09 Worksheet

\begin{ex}
  Recreate Figure~3.1.
\end{ex}

\begin{ex}
  Define the following phrases.
  \begin{enumerate}[(a)]
    \item triangular number
    \item square number
    \item pentagonal number
    \item polygonal number
  \end{enumerate}
\end{ex}

\begin{ex}
  According to the textbook, ``From Figure 3.1 it is easy to calculate an expression for the $m$th $n$-gonal number as the sum of a certain arithmetic series ...'' Do this for:
  \begin{enumerate}[(a)]
    \item the triangular numbers;
    \item the square numbers; and,
    \item the pentagonal numbers.
  \end{enumerate}
\end{ex}

\begin{ex} [3.2.1]
  Show that any square leaves remainder $0$, $1$, or $4$ on division by $8$.
\end{ex}

\begin{ex} [3.2.2]
  Deduce that a sum of three squares leaves remainder $0$, $1$, $2$, $3$, $4$, $5$, or $6$ on division by $8$.
\end{ex}

\textbf{Bonus Material}

For each nonnegative integer $k$, let $P_k(t)$ be the polynomial with rational coefficients given by
\[
  P_0(t)=1,\text{ and }P_k(t)=\binom{t}{k}=\frac{t(t-1)\cdots(t-k+1)}{k!}\text{ for }k\geq1.
\]

\begin{ex}
  Find the following.
  \begin{enumerate}[(a)]
    \item $P_1(t)$
    \item $P_2(t)$
    \item $P_3(t)$
  \end{enumerate}
\end{ex}
\pagebreak
\begin{ex}
  Complete the following table.
  \begin{center}
    \begin{tabular}{| c | c | c | c | c |} \hline
      $t$ & $P_0(t)$ & $P_1(t)$ & $P_2(t)$ & $P_3(t)$ \\ \hline
      $0$ &          &          &          &          \\ \hline
      $1$ &          &          &          &          \\ \hline
      $2$ &          &          &          &          \\ \hline
      $3$ &          &          &          &          \\ \hline
      $4$ &          &          &          &          \\ \hline
      $5$ &          &          &          &          \\ \hline
    \end{tabular}
  \end{center}
\end{ex}

Given a(n integer valued) polynomial $f(t)$ define a new integer-valued polynomial $(\Delta f)(t)$ by
\[
  (\Delta f)(t)=f(t+1)-f(t).
\]

\begin{ex}
  Find the following.
  \begin{enumerate}[(a)]
    \item $(\Delta P_0)(t)$
    \item $(\Delta P_1)(t)$
    \item $(\Delta P_2)(t)$
    \item $(\Delta P_3)(t)$
  \end{enumerate}
\end{ex}

Think of the operator $\Delta$ as a discrete analog of differentiation. We write $\Delta^2 f$ for $\Delta(\Delta f)$.

\begin{ex}
  Let $a$, $b$, and $c$ be integers. And, let $f(t)=aP_0(t)+bP_1(t)+cP_2(t)$. Complete the following table.
  \begin{center}
    \begin{tabular}{| c | c | c | c |} \hline
      $t$ & $f(t)$   & $(\Delta f)(t)$ & $(\Delta^2 f)(t)$ \\ \hline
      $0$ &          &                 &                   \\ \hline
      $1$ &          &                 &                   \\ \hline
      $2$ &          &                 &                   \\ \hline
      $3$ &          &                 &                   \\ \hline
      $4$ &          &                 &                   \\ \hline
      $5$ &          &                 &                   \\ \hline
    \end{tabular}
  \end{center}
\end{ex}

\begin{ex}
  Suppose you are given a sequence of integers $a_0,a_1,\ldots$ and you are asked whether that sequence is of the form $a_n=f(n)$ for some quadratic polynomial (with rational coefficients), how might you test that idea? And, how you find $f$ when such a polynomial exists?
\end{ex}

\end{document}

