\documentclass[12pt]{article}
\usepackage{amscd,amssymb,amsthm,amsxtra,exscale,latexsym,verbatim,paralist}
\usepackage{mathrsfs}
\usepackage[T1]{fontenc}
\usepackage{newtxmath,newtxtext}
\usepackage[margin=1in]{geometry}

%\usepackage{mathtools}
%\usepackage{multicol}
\usepackage{tikz}

\pagestyle{empty} 
\setlength{\parindent}{0pt} 
\setlength{\parskip}{\baselineskip}

\theoremstyle{plain}
\newtheorem{ex}{Exercise}

\renewcommand{\proofname}{Solution}

%\makeatletter
%\renewcommand*\env@matrix[1][*\c@MaxMatrixCols c]{%
%  \hskip -\arraycolsep
%  \let\@ifnextchar\new@ifnextchar
%  \array{#1}}
%\makeatother

\begin{document}

MTH 385 \qquad 2022-02-14 Worksheet

\begin{ex}
  Define \emph{greatest common divisor}.
\end{ex}

\begin{ex}
  Define \emph{relatively prime}.
\end{ex}

\begin{ex}
  Use the Euclidean Algorithm \emph{as presented in the textbook} to compute $\gcd(1001,65)$.
\end{ex}

\begin{ex}
  Based on your experience with Exercise~1, suggest a refinement to the algorithm.
\end{ex}

\begin{ex}
  Find integers $m$ and $n$ such that $\gcd(1001,65)=1001m+65n$.
\end{ex}

\begin{ex}
  Find a counterexample to the following statement: If $n$ is a number that divides $ab$, then $n$ divides $a$ or $b$.
\end{ex}

\begin{ex}
  Prove:  If $n$ is relatively prime to $a$ and $n$ divides $ab$, then $n$ divides $b$.
\end{ex}

\begin{ex}
  State and prove the \emph{Fundamental Theorem of Arithmetic}.
\end{ex}

\begin{ex}
  State the \emph{Well-Ordering Principle}.
\end{ex}

\begin{ex} [3.3.1]
  Use the prime divisor property to show that the proper divisors of $2^{n-1}p$, for any odd prime $p$, are $1,2,2^2,\ldots,2^{n-1}$ and $p,2p,2^2p,\ldots,2^{n-2}p$.
\end{ex}

\begin{ex} [3.3.4]
  The equation $12x+15y=1$ has no integer solution. Why?
\end{ex}

\end{document}

