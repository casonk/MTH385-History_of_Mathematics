\documentclass[12pt]{article}
\usepackage{amscd,amssymb,amsthm,amsxtra,exscale,latexsym,verbatim,paralist}
\usepackage{mathrsfs}
\usepackage[T1]{fontenc}
\usepackage{newtxmath,newtxtext}
\usepackage[margin=1in]{geometry}

%\usepackage{mathtools}
%\usepackage{multicol}
\usepackage{tikz}

\pagestyle{empty} 
\setlength{\parindent}{0pt} 
\setlength{\parskip}{\baselineskip}

\theoremstyle{plain}
\newtheorem{ex}{Exercise}

\renewcommand{\proofname}{Solution}

%\makeatletter
%\renewcommand*\env@matrix[1][*\c@MaxMatrixCols c]{%
%  \hskip -\arraycolsep
%  \let\@ifnextchar\new@ifnextchar
%  \array{#1}}
%\makeatother

\begin{document}

MTH 385 \qquad 2022-02-21 Worksheet

\begin{ex}
  What do we mean when we say a point in $\mathbb{R}^2$ is \emph{rational}?
\end{ex}

\begin{ex}
  How does one find (algebraically) the intersection(s) of a curve $p(x,y)=0$ with a line $y=mx+b$?
\end{ex}

\begin{ex}
  If $p(x,y)$ is a degree $d$ polynomial, how many intersections do we expect the curve $p(x,y)=0$ to have with a line $y=mx+b$?
\end{ex}

\begin{ex}
  What is the \emph{chord method}? What is the goal of the method?
\end{ex}

\begin{ex}
  Suppose $q(x)$ is a polynomial with rational coefficients. Further suppose $q(x)=k(x-r_1)(x-r_2)$ and $r_1$ is rational. Prove $k$ and $r_2$ are also rational.
\end{ex}

\begin{ex}
  Prove: If we know two rational points on a cubic curve $p(x,y)=0$, then a third intersection point on the line through them will also be rational.
\end{ex}

\begin{ex}
  What is the \emph{tangent method}? What is the goal of the method? When can it be used?
\end{ex}

\begin{ex}
  Consider the curve $x^3-y=0$ and the line $y=3x-2$. Check that they are tangent at $(1,1)$. Apply the tangent method.
\end{ex}

\begin{ex}
  If the curve $p(x,y)=0$ has degree $3$ and the line $y=mx+b$ is tangent to the curve at some point $(x_0,y_0)$, what can we expect from the solutions to $p(x,mx+b)=0$?
\end{ex}

\end{document}

