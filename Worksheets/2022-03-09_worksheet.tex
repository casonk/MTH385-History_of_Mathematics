\documentclass[12pt]{article}
\usepackage{amscd,amssymb,amsthm,amsxtra,exscale,latexsym,verbatim,paralist}
\usepackage{mathrsfs}
\usepackage[T1]{fontenc}
\usepackage{newtxmath,newtxtext}
\usepackage[margin=1in]{geometry}

%\usepackage{mathtools}
%\usepackage{multicol}
\usepackage{tikz}

\pagestyle{empty} 
\setlength{\parindent}{0pt} 
\setlength{\parskip}{\baselineskip}

\theoremstyle{plain}
\newtheorem{ex}{Exercise}

\renewcommand{\proofname}{Solution}

%\makeatletter
%\renewcommand*\env@matrix[1][*\c@MaxMatrixCols c]{%
%  \hskip -\arraycolsep
%  \let\@ifnextchar\new@ifnextchar
%  \array{#1}}
%\makeatother

\begin{document}

MTH 385 \qquad 2022-03-09 Worksheet

An elementary proof that $\sqrt[3]{2}$ is not constructible was found by the number theorist Edmund Landau (1877–1938) when he was still a student. It is broken down to easy steps below. But first we should check that $\sqrt[3]{2}$ is actually irrational.

\begin{ex}[5.4.1]
  Show that the assumption $\sqrt[3]{2}=m/n$, where $m$ and $n$ are integers, leads to a contradiction.
\end{ex}

Landau’s proof now organizes all numbers involved in a construction into sets $F_0,F_1,\ldots$, according to the depth of nesting of square roots.

\begin{ex}[5.4.2]
  Let
  \[
    F_0=\{rationals\},\qquad F_{k+1}=\{a+b\sqrt{c_k}\mid a,b,c_k\in F_k\}\text{ for some }c_k\in F_k.
  \]
  Show that each $F_k$ is a \emph{field}, that is, if $x$, $y$ are in $F_k$, then so are $x+y$, $x-y$, $xy$, and $x/y$ (for $y\neq0$).
\end{ex}

\begin{ex}
  Consider the situation of Exercise~5.3.5. Given the equations of two circles in the form
  \begin{align*}
    (x-a)^2+(y-b)^2 &= r^2 \\
    (x-c)^2+(y-d)^2 &= s^2,
  \end{align*}
  When do we get the the equation of a line by subtracting one of the equations from the other? When we get the equation of a line, how is the line related to the circles?
\end{ex}

\end{document}

