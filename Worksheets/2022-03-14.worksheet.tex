\documentclass[12pt]{article}
\usepackage{amscd,amssymb,amsthm,amsxtra,exscale,latexsym,verbatim,paralist}
\usepackage{mathrsfs}
\usepackage[T1]{fontenc}
\usepackage{newtxmath,newtxtext}
\usepackage[margin=1in]{geometry}

%\usepackage{mathtools}
%\usepackage{multicol}
\usepackage{tikz}

\pagestyle{empty} 
\setlength{\parindent}{0pt} 
\setlength{\parskip}{\baselineskip}

\theoremstyle{plain}
\newtheorem{ex}{Exercise}

\renewcommand{\proofname}{Solution}

%\makeatletter
%\renewcommand*\env@matrix[1][*\c@MaxMatrixCols c]{%
%  \hskip -\arraycolsep
%  \let\@ifnextchar\new@ifnextchar
%  \array{#1}}
%\makeatother

\begin{document}

MTH 385 \qquad 2022-03-14 Worksheet

Here, we walk through Cardano's method for find a solution to the (general) cubic equation $x^3+ax^2+bx+c=0$.

\begin{ex}
  Substitute $x=y-a/3$ into $x^3+ax^2+bx+c=0$. Rewrite the equation in the form
  \[
    y^3=py+q.
  \]
  Express $p$ and $q$ in terms of $a$, $b$ and $c$.
\end{ex}

\begin{ex}
  Substitute $y=u+v$ into $y^3$. By collecting terms, rewrite the expression in the form
  \[
    y^3=py+q.
  \]
  Express $p$ and $q$ in terms of $u$, and $v$.
\end{ex}

\begin{ex}
  Use the expression from the previous exercise for $p$ to eliminate $v$ from the expression for $q$ (also from the previous exercise).
\end{ex}

\begin{ex}
  Write and solve the quadratic in $u^3$ obtained from the previous exercise. This gives two expressions for $u^3$.
\end{ex}

\begin{ex}
  Find expressions for $v^3$ similar to the ones for $u^3$ found in the previous exercise.
\end{ex}

\begin{ex}
  We now have four expressions that might be $u^3+v^3$. What are they? Use your knowledge of $u^3+v^3$ from above to choose a valid expression.
\end{ex}

\begin{ex}
  Assume our expressions for $u^3$ and $v^3$ are real numbers. Find real values for $u$ and $v$. Then, find $y$.
\end{ex}

\begin{ex}
  Find a solution to $x^3+ax^2+bx+c=0$.
\end{ex}

The two equations $3uv=p$, $u^3+v^3=q$ provide another instance of the phenomenon noted in Exercise~5.2.2: when a variable is eliminated between two equations, the degrees of the equations are multiplied.

\begin{ex}[5.5.1]
  The equation $3uv=p$ is of degree $2$ in $u$ and $v$, and $u^3+v^3=q$ is of degree $3$. What about the equation obtained by eliminating $v$?
\end{ex}

\end{document}

