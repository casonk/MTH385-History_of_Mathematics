\documentclass[12pt]{article}
\usepackage{amscd,amssymb,amsthm,amsxtra,exscale,latexsym,verbatim,paralist}
\usepackage{mathrsfs}
\usepackage[T1]{fontenc}
\usepackage{newtxmath,newtxtext}
\usepackage[margin=1in]{geometry}

%\usepackage{mathtools}
%\usepackage{multicol}
\usepackage{tikz}

\pagestyle{empty} 
\setlength{\parindent}{0pt} 
\setlength{\parskip}{\baselineskip}

\theoremstyle{plain}
\newtheorem{ex}{Exercise}

\renewcommand{\proofname}{Solution}

%\makeatletter
%\renewcommand*\env@matrix[1][*\c@MaxMatrixCols c]{%
%  \hskip -\arraycolsep
%  \let\@ifnextchar\new@ifnextchar
%  \array{#1}}
%\makeatother

\begin{document}

MTH 385 \qquad 2022-03-16 Worksheet

\begin{ex}
  Check that the equation
  \[
    x^3+ax+b=0
  \]
  can be reduced to the equation
  \[
    4y^3-3y=c
  \]
  by setting $x=ky$ and choosing $k$ so that
  \[
    \frac{k^3}{ak}=-\frac{4}{3}.
  \]
  Express $c$ in terms of $b$, $a$ and $k$.
\end{ex}

\begin{ex}
  Recall the formulas for the sine ane cosine of the sum of two angles.
  \begin{align*}
    \cos(\alpha+\beta) &= \\
    \sin(\alpha+\beta) &=
  \end{align*}
\end{ex}

\begin{ex}
  Verify.
  \[
    4\cos^3\theta-3\cos\theta=\cos3\theta
  \]
\end{ex}

\begin{ex}
  Explain why solving the cubic equation
  \[
    x^3+ax+b=0
  \]
  is equivalent to the problem of trisecting an angle (if $|c|\leq1$).
\end{ex}

\begin{ex}
  Check that Cardano's solution to the cubic does not envolve complex numbers when $|c|>1$.
\end{ex}

\begin{ex}
  Use induction to prove de~Moivre's formula. That is, prove that for all positive integers $n$ and all angles $\theta$,
  \[
    (\cos\theta+i\sin\theta)^n=\cos n\theta+i\sin n\theta.
  \]
\end{ex}

\begin{ex} [a variant of 5.6.1]
  Revisit Exercise~3 in light of de~Moivre's formula.
\end{ex}

\begin{ex} 
  Recall the Maclaurin series (Taylor series at $z=0$) for $e^z$.
\end{ex}

\begin{ex} 
  Find the real and imaginary parts of the Maclaurin series for $e^{ix}$.
\end{ex}

\begin{ex}
  Comment upon the results of Exercise~9.
\end{ex}

\end{document}

