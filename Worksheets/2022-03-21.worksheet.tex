\documentclass[12pt]{article}
\usepackage{amscd,amssymb,amsthm,amsxtra,exscale,latexsym,verbatim,paralist}
\usepackage{mathrsfs}
\usepackage[T1]{fontenc}
\usepackage{newtxmath,newtxtext}
\usepackage[margin=1in]{geometry}

%\usepackage{mathtools}
%\usepackage{multicol}
\usepackage{tikz}

\pagestyle{empty} 
\setlength{\parindent}{0pt} 
\setlength{\parskip}{\baselineskip}

\theoremstyle{plain}
\newtheorem{ex}{Exercise}

\renewcommand{\proofname}{Solution}

%\makeatletter
%\renewcommand*\env@matrix[1][*\c@MaxMatrixCols c]{%
%  \hskip -\arraycolsep
%  \let\@ifnextchar\new@ifnextchar
%  \array{#1}}
%\makeatother

\begin{document}

MTH 385 \qquad 2022-03-21 Worksheet

\begin{ex}
  According to the textbook, the general \emph{quartic} equation
  \[
    x^4+ax^3+bx^2+cx+d=0
  \]
  can be reduced to an equation of the form
  \[
    x^4+px^2+qx+r=0
  \]
  using a linear transformation. Carry out this reduction.
\end{ex}

\begin{ex}
  Show that the equation
  \[
    x^4+px^2+qx+r=0
  \]
  can be rewritten
  \[
    (x^2+p)^2=px^2-qx+p^2-r.
  \]
\end{ex}

\begin{ex}
  Deduce from the previous exercise that
  \[
    (x^2+p+y)^2=(p+2y)x^2-qx+(p^2-r+2py+y^2).
  \]
\end{ex}

\begin{ex}
  Give a criterion for determining when
  \[
    Ax^2+Bx+C
  \]
  is a square.
\end{ex}

\begin{ex}
  Apply the criterion from the previous exercise to determine when the equation
  \[
    (x^2+p+y)^2=(p+2y)x^2-qx+(p^2-r+2py+y^2)
  \]
  has a solution.
\end{ex}

\begin{ex}
  The answer to the previous exercise is a polynomial equation in $y$. What is its degree?
\end{ex}

\begin{ex}
  Outline a procedure for solving a general quartic equation.
\end{ex}

\begin{ex}
  According to the textbook, does a general \emph{quintic} equation have a \emph{solution by radicals}?
\end{ex}

\begin{ex}
  State Descartes's Theorem.
\end{ex}

\begin{ex} [5.7.1]
  Show that $x^n-a^n$ has a factor $x-a$. What is the quotient $(x^n-a^n)/(x-a)$? (And what does this have to do with geometric series?)
\end{ex}

\end{document}

