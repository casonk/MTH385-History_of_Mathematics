\documentclass[12pt]{article}
\usepackage{amscd,amssymb,amsthm,amsxtra,exscale,latexsym,verbatim,paralist}
\usepackage{mathrsfs}
\usepackage[T1]{fontenc}
\usepackage{newtxmath,newtxtext}
\usepackage[margin=1in]{geometry}

%\usepackage{mathtools}
%\usepackage{multicol}
\usepackage{tikz}

\pagestyle{empty} 
\setlength{\parindent}{0pt} 
\setlength{\parskip}{\baselineskip}

\theoremstyle{plain}
\newtheorem{ex}{Exercise}

\renewcommand{\proofname}{Solution}

%\makeatletter
%\renewcommand*\env@matrix[1][*\c@MaxMatrixCols c]{%
%  \hskip -\arraycolsep
%  \let\@ifnextchar\new@ifnextchar
%  \array{#1}}
%\makeatother

\begin{document}

MTH 385 \qquad 2022-03-23 Worksheet

\begin{ex}
  Compute.
  \begin{align*}
    & (a+b)^0 = \hspace{5in} \\
    & (a+b)^1 =  \\
    & (a+b)^2 =  \\
    & (a+b)^3 =  \\
    & (a+b)^4 =  \\
    & (a+b)^5 =  \\
    & (a+b)^6 =  \\
    & (a+b)^7 =
  \end{align*}
\end{ex}

\begin{ex}
  Make a table of the coefficients of the polynomials (when written in decreasing $a$-degree) from the previous exercise.
\end{ex}

Write $\binom{n}{k}$ for the $k$th element of the $n$th row (indexing starting with $0$). That is, $\binom{n}{k}$ is the coefficient on $a^{n-k}b^k$ in $(a+b)^n$. These numbers are called \emph{binomial coefficients}.

\begin{ex} [5.8.1]
  Use the identity
  \[
    (a+b)^n=(a+b)^{n-1}a+(a+b)^{n-1}b
  \]
  to prove the sum property of binomial coefficients:
  \[
    \binom{n}{k}=\binom{n-1}{k-1}+\binom{n-1}{k}.
  \]
\end{ex}

\begin{ex}
  Prove $\binom{n}{k}$ is the number of combinations of $n$ things taken $k$ at a time. That is, prove $\binom{n}{k}$ is the number of $k$-element subsets of an $n$-element set.
\end{ex}

\begin{ex}
  Prove $n!$ is the number of permutations of an $n$-element set.
\end{ex}

\begin{ex}
  Prove
  \[
    \binom{n}{k}=\frac{n!}{(n-k)!k!}.
  \]
\end{ex}

\begin{ex}
  Prove
  \[
    \binom{n}{k+1}=\binom{n}{k}\frac{n-k}{k+1}.
  \]
\end{ex}

\begin{ex}
  Prove
  \[
    \binom{r}{n}\binom{n}{k}=\binom{r}{k}\binom{r-k}{n-k}.
  \]
\end{ex}

\end{document}

