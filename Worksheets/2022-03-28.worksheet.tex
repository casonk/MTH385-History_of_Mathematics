\documentclass[12pt]{article}
\usepackage{amscd,amssymb,amsthm,amsxtra,exscale,latexsym,verbatim,paralist}
\usepackage{mathrsfs}
\usepackage[T1]{fontenc}
\usepackage{newtxmath,newtxtext}
\usepackage[margin=1in]{geometry}

%\usepackage{mathtools}
%\usepackage{multicol}
\usepackage{tikz}

\pagestyle{empty} 
\setlength{\parindent}{0pt} 
\setlength{\parskip}{\baselineskip}

\theoremstyle{plain}
\newtheorem{ex}{Exercise}

\renewcommand{\proofname}{Solution}

%\makeatletter
%\renewcommand*\env@matrix[1][*\c@MaxMatrixCols c]{%
%  \hskip -\arraycolsep
%  \let\@ifnextchar\new@ifnextchar
%  \array{#1}}
%\makeatother

\begin{document}

MTH 385 \qquad 2022-03-28 Worksheet

\begin{ex}
  Who was Pierre de Fermat and when did he live?
\end{ex}

\begin{ex}
  State Fermat's Little Theorem.
\end{ex}

\begin{ex}
  Let $p$ be prime and let $k$ be an integer strictly between $0$ and $p$. Prove $\displaystyle\binom{p}{k}$ is divisible by $p$.
\end{ex}

\begin{ex}
  Let $p$ be prime. Prove $2^p-2$ is divisible by $p$.
\end{ex}

\begin{ex}
  Define the \emph{multinomial coefficient} $\displaystyle\binom{n}{k_1,k_2,\ldots,k_m}$.
\end{ex}

\begin{ex} [5.9.1]
  Use the result $2^p=(1+1)^p=2+\text{ terms divisible by }p$, and its method of proof, to show that
  \[
    3^p=(2+1)^p=3+\text{ terms divisible by }p.
  \]
\end{ex}

\begin{ex} [5.9.2]
  Build on the idea of Exercise~5.9.1 to show that $n^p-n$ is divisible by $p$ for any positive integer $n$.
\end{ex}

\begin{ex} [5.9.3]
  Observe the terms divisible by $p$ in the first few rows of Pascal's triangle, computed in the previous section.
\end{ex}

\end{document}

