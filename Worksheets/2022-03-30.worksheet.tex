\documentclass[12pt]{article}
\usepackage{amscd,amssymb,amsthm,amsxtra,exscale,latexsym,verbatim,paralist}
\usepackage{mathrsfs}
\usepackage[T1]{fontenc}
\usepackage{newtxmath,newtxtext}
\usepackage[margin=1in]{geometry}

%\usepackage{mathtools}
%\usepackage{multicol}
\usepackage{tikz}

\pagestyle{empty} 
\setlength{\parindent}{0pt} 
\setlength{\parskip}{\baselineskip}

\theoremstyle{plain}
\newtheorem{ex}{Exercise}

\renewcommand{\proofname}{Solution}

%\makeatletter
%\renewcommand*\env@matrix[1][*\c@MaxMatrixCols c]{%
%  \hskip -\arraycolsep
%  \let\@ifnextchar\new@ifnextchar
%  \array{#1}}
%\makeatother

\begin{document}

MTH 385 \qquad 2022-03-30 Worksheet

\begin{ex}
  Recall the idea of Menaechmus that gives ``a very simple solution to the problem of duplicating the cube'' from Section~2.4 of the textbook.
\end{ex}

\begin{ex} [6.1.1]
  Generalize the idea of Menaechmus to show that any cubic equation
  \[
    ax^3+bx^2+cx+d=0\qquad\text{ and}\qquad d\neq0
  \]
  may be solved by intersecting the hyperbola $xy=1$ with a parabola.
\end{ex}

\begin{ex}
  In the $17$th century, Fermat and Descartes developed an important mathematical innovation. What was it?
\end{ex}

\begin{ex}
  What then-unsolved problem does the textbook use to illustrate Descartes' hubris? What did he say about it? And, how is the problem solved?
\end{ex}

\end{document}

