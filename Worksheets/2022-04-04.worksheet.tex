\documentclass[12pt]{article}
\usepackage{amscd,amssymb,amsthm,amsxtra,exscale,latexsym,verbatim,paralist}
\usepackage{mathrsfs}
\usepackage[T1]{fontenc}
\usepackage{newtxmath,newtxtext}
\usepackage[margin=1in]{geometry}

%\usepackage{mathtools}
%\usepackage{multicol}
\usepackage{tikz}

\pagestyle{empty} 
\setlength{\parindent}{0pt} 
\setlength{\parskip}{\baselineskip}

\theoremstyle{plain}
\newtheorem{ex}{Exercise}

\renewcommand{\proofname}{Solution}

%\makeatletter
%\renewcommand*\env@matrix[1][*\c@MaxMatrixCols c]{%
%  \hskip -\arraycolsep
%  \let\@ifnextchar\new@ifnextchar
%  \array{#1}}
%\makeatother

\begin{document}

MTH 385 \qquad 2022-04-04 Worksheet

\begin{ex}
  When was calculus developed? And who where the main developers?
\end{ex}

\begin{ex} 
  Which developer used the \emph{binomial theorem} as his starting point? And, how is his approach reflected today?
\end{ex}

\begin{ex} 
  Which developer used the \emph{infinitesimals} as his starting point? And, how is his approach reflected today?
\end{ex}

\begin{ex} [8.2.1]
  Find $1+2+\cdots+n$ by summing the identity $(m+1)^2-m^2=2m+1$ from $m=1$ to $n$. Similarly find $1^2+2^2+\cdots+n^2$ using the identity
  \[
    (m+1)^3-m^3=3m^2+3m+1
  \]
  together with the previous result. Likewise, find $1^3+2^3+\cdots+n^3$ using the identity
  \[
    (m+1)^4-m^4=4m^3+6m^2+4m+1
  \]
  and so on.
\end{ex}

\begin{center}
  \begin{tikzpicture} [scale=0.9]
    \draw [-] (-1,0) -- (11,0);
    \draw [-] (0,-1) -- (0,11);
    \draw[domain=0:10.5, smooth, variable=\x, red] plot ({\x}, {\x*\x/10});
    \draw [-] (0,0.1) -- (1,0.1);
    \draw [-] (1,0) -- (1,0.4) -- (2,0.4);
    \draw [-] (2,0) -- (2,0.9) -- (3,0.9) -- (3,0);
    \draw [-] (9,0) -- (9,10) -- (10,10) -- (10,0);
    \node [below left] at (0,0) {$0$};
    \node [below] at (1,0) {$\frac{1}{n}$};
    \node [below] at (2,0) {$\frac{2}{n}$};
    \node [below] at (3,0) {$\frac{3}{n}$};
    \node [below] at (6,0) {$\cdots$};
    \node [below] at (9,0) {$\frac{n-1}{n}$};
    \node [below] at (10,0) {$\frac{n}{n}$};
    \node [above] at (10.5,11.025) {$y=x^k$};
  \end{tikzpicture}

  \caption{Figure~8.1: Approximating an area by rectangles}
\end{center}

\begin{ex} [8.2.2]
  Show that the approximation to the area under $y=x^2$ by rectangles in Figure~8.1 has value $(2n+1)n(n+1)/6n^3$, and deduce that the area under the curve is $1/3$.
\end{ex}

\end{document}

