\documentclass[12pt]{article}
\usepackage{amscd,amssymb,amsthm,amsxtra,exscale,latexsym,verbatim,paralist}
\usepackage{mathrsfs}
\usepackage[T1]{fontenc}
\usepackage{newtxmath,newtxtext}
\usepackage[margin=1in]{geometry}

%\usepackage{mathtools}
%\usepackage{multicol}
\usepackage{tikz}

\pagestyle{empty} 
\setlength{\parindent}{0pt} 
\setlength{\parskip}{\baselineskip}

\theoremstyle{plain}
\newtheorem{ex}{Exercise}

\renewcommand{\proofname}{Solution}

%\makeatletter
%\renewcommand*\env@matrix[1][*\c@MaxMatrixCols c]{%
%  \hskip -\arraycolsep
%  \let\@ifnextchar\new@ifnextchar
%  \array{#1}}
%\makeatother

\begin{document}

MTH 385 \qquad 2022-04-06 Worksheet

\begin{ex}
  Historically, which was considered simpler, differentiation or integration?
\end{ex}

\begin{ex}
  What is the earliest example of the limit process
  \[
    \lim_{\Delta x\to0}\frac{f(x+\Delta x)-f(x)}{\Delta x}?
  \]
\end{ex}

\begin{ex}
  When did this limit process appear for polynomials?
\end{ex}

\begin{ex}
  Suppose $\varepsilon^2=0$ and we can divide by $\varepsilon\neq0$. Compute $\displaystyle\frac{f(x+\varepsilon)-f(x)}{\varepsilon}$ for the following polynomials.
  \begin{enumerate}[(a)]
    \item $f(x)=1$
    \item $f(x)=x$
    \item $f(x)=x^2$
    \item $f(x)=x^3$
    \item $f(x)=2x^2-3x+5$
  \end{enumerate}
\end{ex}

\begin{ex}
  Comment on the results of the previous exercise. What was Fermat's method? And, how is it related to the method of the previous exercise?
\end{ex}

\begin{ex}
  Let $p(x,y)$ be a polynomial. And, consider the curve $p(x,y)=0$. What is the formula of Hudde and Sluse for $\dfrac{dy}{dx}$ when
  \[
    p(x,y)=\sum a_{ij}x^iy^j?
  \]
\end{ex}

For evidence that tangents to algebraic curves may be found without calculus, it is enough to look more closely at what we called Diophantus's tangent method in Section~3.5. In his \emph{Arithmetica}, Problem~18, Book~VI (previously mentioned in Exercise~3.5.1), Diophantus finds the tangent $y=\frac{3x}{2}+1$ to $y^2=x^3-3x^2+3x+1$ at the point $(0,1)$, apparently by inspection. Without mentioning its geometric interpretation, he simply substitutes $y=\frac{3x}{2}+1$ for $y$ in $y^2=x^3-3x^2+3x+1$.

\begin{ex} [8.3.1]
  Check that this substitution gives the equation
  \[
    x^3-\frac{21}{4}x^2=0
  \]
  What is the geometric interpretation of the double root $x=0$?
\end{ex}

\begin{ex} [8.3.2]
  What would you substitute for y to find the tangent at $(0,1)$ to the curve $y^2=x^3-3x^2+5x+1$?
\end{ex}

\end{document}

