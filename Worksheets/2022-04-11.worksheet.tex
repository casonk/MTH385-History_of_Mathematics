\documentclass[12pt]{article}
\usepackage{amscd,amssymb,amsthm,amsxtra,exscale,latexsym,verbatim,paralist}
\usepackage{mathrsfs}
\usepackage[T1]{fontenc}
\usepackage{newtxmath,newtxtext}
\usepackage[margin=1in]{geometry}

%\usepackage{mathtools}
%\usepackage{multicol}
\usepackage{tikz}

\pagestyle{empty} 
\setlength{\parindent}{0pt} 
\setlength{\parskip}{\baselineskip}

\theoremstyle{plain}
\newtheorem{ex}{Exercise}

\renewcommand{\proofname}{Solution}

%\makeatletter
%\renewcommand*\env@matrix[1][*\c@MaxMatrixCols c]{%
%  \hskip -\arraycolsep
%  \let\@ifnextchar\new@ifnextchar
%  \array{#1}}
%\makeatother

\begin{document}

MTH 385 \qquad 2022-04-11 Worksheet

\begin{ex}
  The title of Section~8.4 mentions \emph{Arithmetica Infinitorum}. What was \emph{Arithmetica Infinitorum}?
\end{ex}

\begin{ex}
  How did Wallis compute $\int_0^1x^{1/3}\,dx$? Carry out the calculation.
\end{ex}

\begin{ex}
  Repeat Wallis' infinite product formula for $\frac{\pi}{4}$.
\end{ex}

\begin{ex} [8.4.1]
  Use the identity $\sin x=2\sin(x/2)\cos(x/2)$ to show that
  \[
    \frac{\sin x}{2^n\sin(x/2^n)}=\cos\frac{x}{2}\cos\frac{x}{2^2}\cdots\cos\frac{x}{2^n}
  \]
  whence
  \[
    \frac{\sin x}{x}=\cos\frac{x}{2}\cos\frac{x}{2^2}\cos\frac{x}{2^3}\cdots.
  \]
\end{ex}

\begin{ex} [8.4.2]
  Deduce Vi\`{e}te’s product by substituting $x=\pi/2$.
\end{ex}

\end{document}
