\documentclass[12pt]{article}
\usepackage{amscd,amssymb,amsthm,amsxtra,exscale,latexsym,verbatim,paralist}
\usepackage{mathrsfs}
\usepackage[T1]{fontenc}
\usepackage{newtxmath,newtxtext}
\usepackage[margin=1in]{geometry}

%\usepackage{mathtools}
%\usepackage{multicol}
\usepackage{tikz}

\pagestyle{empty} 
\setlength{\parindent}{0pt} 
\setlength{\parskip}{\baselineskip}

\theoremstyle{plain}
\newtheorem{ex}{Exercise}

\renewcommand{\proofname}{Solution}

%\makeatletter
%\renewcommand*\env@matrix[1][*\c@MaxMatrixCols c]{%
%  \hskip -\arraycolsep
%  \let\@ifnextchar\new@ifnextchar
%  \array{#1}}
%\makeatother

\begin{document}

MTH 385 \qquad 2022-04-18 Worksheet

\begin{ex}
  Why was there a dispute over whether Newton discovered calculus?
\end{ex}

\begin{ex}
  What were Leibniz' contributions to calculus? How did he think about calculus?
\end{ex}

\begin{ex}
  Who introduced the word ``function''?
\end{ex}

\begin{ex}
  What is an algebraic function? What is a transcendental function? How do they differ?
\end{ex}

\begin{ex}
  What is a ``closed-form'' expression?
\end{ex}

\begin{ex}
  According to the textbook, ``the search for closed forms was a wild goose chase.'' Give an example of an algebraic function that does not have an algebraic anti-derivative.
\end{ex}

\begin{ex}
  Sketch a proof of the Fundamental Theorem  of Calculus.
\end{ex}

Leibniz~(1702) was stymied by the integral $\int\frac{dx}{x^4+1}$, because he did not spot the factorization of $x^4+1$ into real quadratic factors.

\begin{ex} [8.6.1]
  Writing $x^4+1=x^4+2x^2+1-2x^2$ or otherwise, split $x^4+1$ into real quadratic factors.
\end{ex}

\begin{ex} [8.6.2]
  Use the factors in Exercise~8.6.1 to express $\frac{1}{x^4+1}$ in the partial fraction form
  \[
    \frac{x+\sqrt{2}}{q_1(x)}+\frac{x-\sqrt{2}}{q_2(x)}
  \]
  where $q_1(x)$ and $q_2(x)$ are real quadratic polynomials.
\end{ex}

\begin{ex} [8.6.3]
  Without working out all the details, explain how the partial fractions in Exercise~8.6.2 can be integrated in terms of rational functions and the $\tan^{-1}$ function.
\end{ex}

\end{document}
