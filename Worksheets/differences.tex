\documentclass[12pt]{article}
\usepackage{amscd,amssymb,amsthm,amsxtra,exscale,latexsym,verbatim,paralist}
\usepackage{mathrsfs}
\usepackage[T1]{fontenc}
\usepackage{newtxmath,newtxtext}
\usepackage[margin=1in]{geometry}

%\usepackage{mathtools}
%\usepackage{multicol}
\usepackage{tikz}

\pagestyle{empty} 
\setlength{\parindent}{0pt} 
\setlength{\parskip}{\baselineskip}

\theoremstyle{plain}
\newtheorem{ex}{Exercise}

\renewcommand{\proofname}{Answer}

%\makeatletter
%\renewcommand*\env@matrix[1][*\c@MaxMatrixCols c]{%
%  \hskip -\arraycolsep
%  \let\@ifnextchar\new@ifnextchar
%  \array{#1}}
%\makeatother

\begin{document}

Comments Related to Binomial Coefficients, Differences, and Integer-Valued Polynomials \vspace{\baselineskip}

The question addressed by these comments is: Given a sequence of real numbers
\[
  (a_n)_{n=0}^\infty=(a_0,a_1,a_2,\ldots)
\]
can we find a polynomial $f(x)$ such that $a_n=f(n)$ for all $n$?

[We will use a technique inspired by the study of integer-valued polynomials. So, taking $(a_n)_{n=0}^\infty$ to be integers is fine.]

\begin{ex}
  Let $f(x)=2x^2-3x+5$. Find the first five terms of the sequence $(f(n))_{n=0}^\infty$.
\end{ex}

Let $f(x)$ be a polynomial. Define a new polynomial $\Delta f(x)$ by 
\[
  \Delta f(x)=f(x+1)-f(x).
\]

We write $\Delta^2f(x)=\Delta(\Delta f(x))$, $\Delta^3f(x)=\Delta(\Delta^2f(x))$, and so on.

\begin{ex}
  Find $\Delta(2x^2-3x+5)$ and $\Delta^2(2x^2-3x+5)$.
\end{ex}

\begin{ex}
  Check that if $f(x)$ is a polynomial of positive degree $d$, $\Delta f(x)$ is a polynomial of degree $d-1$.
\end{ex}

\begin{ex}
  Check that if $f(x)$ is a polynomial of degree $d$, $\Delta^{d+1}f(x)=0$.
\end{ex}

\begin{ex}
  Check that if $f(x)$ and $g(x)$ are polynomials and $c$ is a real number, then
  \begin{enumerate}[1.]
    \item $\Delta(f(x)+g(x))=\Delta f(x)+\Delta g(x)$;
    \item $\Delta cf(x)=c\Delta f(x)$.
  \end{enumerate}
\end{ex}

Define the polynomials $p_k(x)$, for $k=0,1,2,\ldots$ by $p_0(x)=1$ and
\[
  p_k(x)=\binom{x}{k}=\frac{x(x-1)(x-2)\cdots(x-k+1)}{k!}
\]
for $k>1$. Notice that if $n$ is a nonnegative integer then
\[
  p_k(n)=
  \begin{cases}
    0,            & \text{for } 0\leq n\leq k-1 \\
    \binom{n}{k}, & \text{for } n\geq k
  \end{cases}.
\]
People sometimes write $p_k(x)=\binom{x}{k}$.

\begin{ex}
  Check that if $\Delta p_k(x)$ is $p_{k-1}(x)$.
\end{ex}

Given a sequence $(a_n)_{n=0}^\infty$, define a new sequence $(b_n)_{n=0}^\infty=\Delta(a_n)_{n=0}^\infty$ by
\[
  b_n=a_{n+1}-a_n
\]

We write $\Delta^2(a_n)_{n=0}^\infty=\Delta(\Delta(a_n)_{n=0}^\infty)$, $\Delta^3(a_n)_{n=0}^\infty=\Delta(\Delta^2(a_n)_{n=0}^\infty)$, and so on.

\begin{ex}
  Check that $\Delta(f(n))_{n=0}^\infty=(\Delta f(n))_{n=0}^\infty$. That is the difference of a sequence given by a polynomial is given by the difference of the polynomial.
\end{ex}

\begin{ex}
  Compute
  \begin{enumerate}[(a)]
    \item the first five terms of the sequence $\Delta(0,1,4,10,20,35,\ldots)$;
    \item the first four terms of the sequence $\Delta^2(0,1,4,10,20,35,\ldots)$;
    \item the first three terms of the sequence $\Delta^3(0,1,4,10,20,35,\ldots)$;
    \item the first two terms of the sequence $\Delta^4(0,1,4,10,20,35,\ldots)$.
  \end{enumerate}
\end{ex}

Assume $(0,1,4,10,20,35,\ldots)$ is given by a polynomial $f(x)$, then the degree $f(x)$ is at least $3$.

\begin{ex}
  Assume $(0,1,4,10,20,35,\ldots)$ is given by a polynomial $f(x)$. What is the minimum possible degree of $f(x)$?
\end{ex}

Assume $f(x)$ has that minimum degree.

\begin{ex}
  Check the following.
  \begin{enumerate}[(a)]
    \item $\Delta^3f(x)=p_0(x)$
    \item $\Delta^2f(x)=p_1(x)+2p_0(x)$
    \item $\Delta f(x)=p_2(x)+2p_1(x)+p_0(x)$
    \item $f(x)=p_3(x)+2p_2(x)+p_1(x)+0p_0(x)$
  \end{enumerate}
\end{ex}

\begin{ex}
  How are the coefficients on the $p_k(x)$ in the previous exercise determined?
\end{ex}

Notice that, given a sequence $(a_n)_{n=0}^\infty$ and a positive integer $d$, this procedure gives a way of finding a polynomial $f(x)$ of degree at most $d$ such that $a_n=f(n)$ when such a polynomial exists. Furthermore, $f(x)$ is a linear combination of the $p_k(x)$ when the sequence consists of integers.

\end{document}

