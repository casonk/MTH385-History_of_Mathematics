\documentclass[12pt]{article}
\usepackage{amscd,amssymb,amsthm,amsxtra,exscale,latexsym,verbatim,paralist}
\usepackage{mathrsfs}
\usepackage[T1]{fontenc}
\usepackage{newtxmath,newtxtext}
\usepackage[margin=1in]{geometry}

%\usepackage{mathtools}
%\usepackage{multicol}
\usepackage{tikz}

\pagestyle{empty} 
\setlength{\parindent}{0pt} 
\setlength{\parskip}{\baselineskip}

\theoremstyle{plain}
\newtheorem{ex}{Exercise}
\newtheorem*{ex8}{Exercise 8}

\renewcommand{\proofname}{Solution}

%\makeatletter
%\renewcommand*\env@matrix[1][*\c@MaxMatrixCols c]{%
%  \hskip -\arraycolsep
%  \let\@ifnextchar\new@ifnextchar
%  \array{#1}}
%\makeatother

\begin{document}

\begin{ex8} [8.3.2]
  What would you substitute for $y$ to find the tangent at $(0,1)$ to the curve $y^2=x^3-3x^2+5x+1$?
\end{ex8}

\begin{proof}
  The equation of the line through $(0,1)$ with slope $m$ is $y=mx+1$. We will substitute $y=mx+1$ and look to see which values of $m$ yield a double root at $x=0$.
  \begin{align*}
    (mx+1)^2      &= x^3-3x^2+5x+1 \\
    m^2x^x+2mx+1  &= x^3-3x^2+5x+1 \\
    0             &= x^3-(m^2+3)x^2+(5-2m)x
  \end{align*}
  Evidently, $y=mx+1$ is tangent at $(0,1)$ to the curve $y^2=x^3-3x^2+5x+1$ exactly when $5-2m=0$. So, the tangent line is $y=\frac{5}{2}x+1$.
\end{proof}

\end{document}

